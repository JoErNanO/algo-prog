\documentclass[10pt]{article}\usepackage[correction,nu]{esial}
%\documentclass[10pt]{article}\usepackage[nu]{esial}
\usepackage{amstext,amsmath,amsfonts}
\TOP\1A
\newcommand{\WP}[1]{\textbf{WP}($#1$)}

\usepackage{amsthm,pifont,textcomp}
\usepackage{amsmath,amssymb}

\usepackage[utf8]{inputenc}
\graphicspath{{fig/}}
\newcommand{\bareme}[1]{{\small (#1 pt)}}

\begin{document}
\title{Examen du 17/06/2010 (2h -- conso)}
\fvset{fontsize=\footnotesize}
\maketitle

\begin{centering}
  \textbf{\large Documents interdits, à l'exception d'une feuille A4 à rendre
    avec votre copie.}

\end{centering}
\centerline{La notation tiendra compte de la présentation et de la clarté de
  la rédaction.}
\bigskip



\bigskip\QuestionCours~(4pt)

\Question(2pts) Décrivez en quelques mots chacun des algorithmes suivants: tri
à bulle, tri par sélection, tri fusion et tri par insertion.

\Question(1pt) À quelles classes de complexité (en notation
$\Theta$) appartiennent les algorithmes 1 et 2 suivants?
%

\medskip\noindent\begin{minipage}{.45\linewidth}
  \begin{Verbatim}[label=algorithme 1]
pour i = 1 à n faire
  pour j = 1 à n faire
     x += 3    
  \end{Verbatim}
\end{minipage}\hfill\begin{minipage}{.45\linewidth}
  \begin{Verbatim}[label=algorithme 2]
pour i = 1 à n faire
  pour j = 1 à n faire
    x += 3
pour i = 1 à n faire
  y = x + 5
  \end{Verbatim}
\end{minipage}

\Question(1pt) Qu'est ce que le backtracking?

\begin{Reponse}
  C'est une technique algorithmique permettant d'effectuer des recherches
  combinatoires bien plus efficacement qu'un parcours exaustif. L'idée de base
  est de construire peu à peu les solutions en ne parcourant que des
  sous-solutions valides. Dès qu'une sous-solution est invalide, on arrête
  l'exploration de cette branche, ce qui permet d'éviter le parcours de
  nombreuses solutions complètes qui seraient invalidées par la présence de la
  sous solution courante. Ouf.
  
  Pour mettre en œuvre le backtracking, on utilise généralement une récursion
  (pour construire peu à peu la solution) avec une boucle ou similaire pour
  parcourir tous les choix possibles à une étape donnée de la construction de
  notre solution. Pour chaque choix, s'il mène à une sous-solution valide
  également, on opère un appel récursif pour explorer les sous-solutions plus
  grandes que l'on peut construire avec celle que l'on a pour l'instant. 
\end{Reponse}


\bigskip\Exercice\textbf{Code récursif mystère} (5pt). 

\noindent\begin{minipage}{.56\linewidth}
%On considère l'algorithme présenté à droite.

  \Question(1pt) On considére le code de droite et les tableaux
  \texttt{tab=\{3, 4, 1, 2\}} et \texttt{tab2=\{17, 12, 51, 20\}}.  Explicitez
  les appels \fbox{\texttt{mystification(tab)}} et
  \fbox{\texttt{mystification(tab2)}} en indiquant le détail de chaque appel
  récursif.

\begin{Reponse}
\begin{verbatim}
mystification([3,4,1,2], 0)
  3 > 2 -> return mystification([3,4,1,2], 1)
                4 > 2 -> return 1+mystification([3,4,1,2], 2)   
                              1 > 2  -> return 1+(1+mystification([3,4,1,2], 3))    
                                              1+mystification([3,4,1,2], 4)
                                                           return 1+1+0       
\end{verbatim}
\end{Reponse}

\Question(\textonehalf pt) Que semble calculer ce code?
\begin{Reponse}
Cette fonction calcule le nombre de valeurs contenues dans le tableau et qui
sont strictement plus grande que la dernière valeur du tableau. 
\end{Reponse}

\Question(\textonehalf pt) Montrez la terminaison de la fonction récursive
\texttt{mystification}.
\begin{Reponse}
La fonction ``s'arrête'' quand \texttt{i} est égal à \texttt{tab.length-1}.
Au premier appel, \texttt{i = 0}. Puis à chaque appel la valeur de \texttt{i}
croît strictement. Elle atteindra donc inévitablement la valeur
\texttt{tab.length-1}. 
\end{Reponse}

  
\end{minipage}~\begin{minipage}{.44\linewidth}
  \VerbatimInput[numbers=right]{Mystere.java}
\end{minipage}



\Question(\textonehalf pt) Quelle est la complexité  de
\texttt{mystification} (en nombre d'appels récursifs)? 
\begin{Reponse}
  Le paramètre de récursion, $i$, croit linéairement. Il faut donc naïvement
  $O(n)$ étapes pour traiter un problème de taille $n$.
\end{Reponse}

\Question(\textonehalf pt) Est-il possible de dérécursiver directement cette
fonction ? Pourquoi ? 

\begin{Reponse}
  Non, car elle n'est pas terminale: il y a des calculs à la remontée (en ligne
  10). 
\end{Reponse}

\Question(2pt) Dérécursivez cette fonction en appliquant les méthodes vues en
cours (en une ou plusieurs étapes). Explicitez ce que vous faites et pourquoi.

\begin{Reponse}
  La première étape est de rendre cette fonction récursive terminale. Pour
  cela, on écrit une fonction ayant un argument supplémentaire, et on y fait
  lors de la descente les opérations que l'on aurait dû faire lors de la
  remontée. Ceci n'est bien sûr possible que parce que l'opération à modifier
  est une addition, qui est associative et commutative.

  \begin{Verbatim}
public int mystification(int [ ] tab) {
  return mystification(tab,0,0);
}
public int mystification(int [ ] tab, int i, int accumulateur) {
  if (i==tab.length-1) {
    return accumulateur;
  } else {
    if (tab[i] > tab[tab.length-1]) {
      return mystification(tab, i+1, accumulateur+1);
    } else {
      return mystification(tab, i+1, accumulateur);
    }
  }
}
  \end{Verbatim}

Une fois rendue terminale, on peut dérécursiver cette fonction de la façon
[simple] vue en cours.

  \begin{Verbatim}
public int mystification(int [ ] tab) {
  int accumulateur = 0;
  int i = 0;
  while (i!=tab.length-1) {
    if (tab[i] > tab[tab.length-1]) {
      accumulateur ++;
      i++;
    } else {
      i++;
    }
  }

  return accumulateur
}
  \end{Verbatim}

  On peut évidement factoriser du code et écrire i++ une seule fois.
\end{Reponse}

\bigskip\Exercice\textbf{Encore un code récursif (mais pas mystère
  normalement)} (7pt). 

\noindent La multiplication de deux nombres entiers positifs peut se faire de
la manière suivante:

\noindent\begin{minipage}{.87\linewidth}

\begin{itemize}
\item Écrivez le multiplicateur et le multiplicande l'un à côté de l'autre.
\item Formez une colonne en dessous de chacun des opérandes en itérant la règle
  suivante jusqu'à ce que le nombre sous le multiplicateur soit égal à 1:
  divisez par deux le nombre sous le multiplicateur, sans tenir compte du
  reste éventuel, et doublez par addition le nombre sous le multiplicande.

  Par exemple, pour multiplier 19 par 45, vous obtenez la suite de nombres
  de droite.
\end{itemize}
  
\end{minipage}~\begin{minipage}{.13\linewidth}
  \begin{Verbatim}[numbers=none]
  45    19
  22    38
  11    76
  5     152
  2     304
  1     608    
  \end{Verbatim}
\end{minipage}

\smallskip\noindent\begin{minipage}{\linewidth}
\begin{itemize}
\item Finalement, rayez tous les nombres de la colonne du multiplicande
  correspondant à multiplicateur pair (ceux pour les multiplicateurs 22 et 2).
  Il ne reste plus qu'à additionner les nombres restants :\\ 19 + 76 + 152 +
  608 = 855.
\end{itemize}  
\end{minipage}


\medskip\noindent
Une autre façon de présenter ce calcul est la suivante: 
$$
\begin{array}{l l l}
  45\times 19 &=& (1+44)\times 19 = 19 + 22\times 2\times 19 = 19 + 22 \times 38\\
  &=&19+11\times 2\times 38=19+11\times76\\
  &=&19+(1+5\times 2)\times76=19+76+5\times152\\
  &=&\ldots
\end{array}$$

\Question(3pt) Écrivez un algorithme (récursif) calculant la multiplication de
cette manière.

\Question(1pt) Montrez la terminaison de votre algorithme. Est il récursif
terminal?

\Question(1pt) Calculez la complexité temporelle de votre code (en nombre
d'appels) dans le cas moyen, dans le meilleur cas et dans le pire cas.

\Question(2pt) Dérécursivez votre code.

%%%%%%%%%%%%%%%%%%%%%%%%%%%%



\bigskip\Exercice\textbf{Preuve de programmes} (4pt).

\noindent\begin{minipage}{.73\linewidth}
  Considérez le code de la fonction ci-contre calculant la factorielle de façon
  itérative. Calculez la plus faible précondition (Weakest Precondition,
  \textbf{WP}) nécessaire pour que la post-condition soit:

  \begin{center}
    $Q \triangleq res = n!$
  \end{center}

  Les règles de calcul des préconditions sont rappelées en annexe.
\end{minipage}\hfill\begin{minipage}{.24\linewidth}
\begin{Verbatim}[numbers=right]
int res;
void Factorial (int n) {
  int f = 1;
  int i = 1;
  while (i < n) {
    i = i + 1;
    f = f * i;
  }
  res = f;
}
\end{Verbatim}  
\end{minipage}

\newcommand{\eq}{ \:\triangleq\: }
\newcommand{\ET}{ \:\wedge\: }
\begin{Reponse}
  Il faut, comme toujours, partir du bas de l'algorithme. Calculons tout
  d'abord la WP permettant à la boucle while d'avoir Q en post-condition.
%
  L'invariant de la boucle est assez simple dans ce cas: $I\eq (f=i!)\ET (i\in[0,n])$
  Le variant est $n-i$

  \medskip\noindent 
  WP(while, Q) = I, avec les obligations habituelles. Considérons d'abord les
  deux dernières obligations, habituellement plus simples.

  \medskip\noindent
  \begin{tabbing}
  Obligation 2 \=$\eq I\Rightarrow V\geq 0$\\
  \>$\eq (f=i!)\ET (i\in[0,n])\Rightarrow n-i\geq 0$\\
  \>$\Leftarrow i\in[0,n]\Rightarrow n-i\geq 0$ ~~~(ce qui est trivialement
  vrai)\\
  \end{tabbing}

  \medskip\noindent
  \begin{tabbing}
  Obligation 3 \=$\eq (E=\mathtt{false}\ET I) \Rightarrow Q$\\
  \>$\eq n=i \ET (f=i!)\ET (i\in[0,n]) \Rightarrow f=n!$\\
  \>$\Leftarrow n=i \ET (f=i!) \Rightarrow f=n!$ ~~~ (c'est bon aussi).
  \end{tabbing}
  Reste la première obligation de preuve, la plus dure.

  \medskip\noindent
  \begin{tabbing}
  Obligation 1 \=$\eq (E=\mathtt{true}\ET I\ET V=z) \Rightarrow
    \mathbf{WP}(C,I\ET V<z))$
  \end{tabbing}

  Commençons par calculer le membre droit de l'implication
  \begin{tabbing}
    $\mathbf{WP}(C,I\ET V<z)$\=$\eq \mathbf{WP}(C,(f=i!)\ET
    (i\in[0,n])\ET n-i<z) $\\
    \>$\eq \left((f=i!)\ET (i\in[0,n])\ET n-i<z\right)_{[i:=i+1;f:=f+1]}$\\
    \>$\eq \left((f+1=(i+1)!)\ET (i+1\in[0,n])\ET n-i-1<z\right)$
  \end{tabbing}

  Ce qui nous donne:
  \begin{tabbing}
  Obligation 1 \=$\eq i<n \ET f=i! \ET i\in[0,n] \ET n-i=z$
  \=$\Rightarrow f\times i=(i+1)! \ET i+1\in[0,n] \ET n-i-1<z$\\
  \>$\eq f=i! \ET i\in[0,n[ \ET n-i=z$
  \>$\Rightarrow  f\times i=(i+1)! \ET  i+1\in[0,n] \ET n-i-1<z$  
  \end{tabbing}
  Notons $P_1$, $P_2$ et $P_3$ les trois prédicats à droite de
  l'implication.
  \begin{itemize}
  \item $P_1\eq f\times i=(i+1)!$ est donné par le fait que $f=i!$ (en prémisse
    de l'implication) et la définition de la factorielle.
  \item $P_2\eq i+1\in[0,n]$ se déduit de la prémisse $i\in[0,n[$
  \item $P_3\eq n-i-1<z$ se déduit de la prémisse $n-i=z$
  \end{itemize}

  \hrule

  On a donc montré que $WP(while,Q)\equiv I$. Reste à finir.

  \noindent
  \begin{tabbing}
  WP(l3-4,I)~~\=$\eq$ WP(l3-4,$(f=i!)\ET (i\in[0,n])$)\\    
  \>$\eq\left((f=i!)\ET (i\in[0,n])\right)_{[i:=1,f:=1]}$\\    
  \>$\eq\left((1=1!)\ET (1\in[0,n])\right)$    \\
  \>$\eq n\geq 1$  
  \end{tabbing}
  L'élément de droite est toujours vrai par définition de la factorielle, et
  l'élément de gauche est vrai si et seulement si n est plus grand que 1
  (sinon, l'écriture est même invalide).

  Donc c'est fini. \fbox{WP(factoriel, res=n!)$\equiv n\geq 1$}
\end{Reponse}

\bigskip
\noindent\hspace{-1.3em}$\bigstar$\textbf{Annexe:} Règles de calcul des
préconditions. 

\begin{itemize}
\item \WP{nop, Q}  $\equiv Q$
\item \WP{x:=E, Q} $\equiv Q[x:=E]$
\item \WP{C;D, Q}  $\equiv$ \WP{C, \WP{D,Q}}
\item \textbf{WP}(\texttt{if} $Cond$ \texttt{then} $C$ \texttt{else} $D$,$Q$)
  $\equiv (Cond=\mathtt{true}\Rightarrow \mathbf{WP}(C,Q))~\wedge~
          (Cond=\mathtt{false}\Rightarrow \mathbf{WP}(D,Q))$
\item \textbf{WP}(\texttt{while} $E$ \texttt{do} $C$ \texttt{done} \{inv I var
  V\},Q)  $\equiv I$ ~~;~~  Obligations de preuve:
  \begin{enumerate}
  \item[$\bullet$] $(E=\mathtt{true}\wedge I\wedge V=z) \Rightarrow
    \mathbf{WP}(C,I\wedge V<z))$
  \item[$\bullet$] $I\Rightarrow V\geq 0$
  \item[$\bullet$] $(E=\mathtt{false}\wedge I) \Rightarrow Q$
  \end{enumerate}
\end{itemize}


\end{document}


%%% Local Variables:
%%% coding: utf-8
