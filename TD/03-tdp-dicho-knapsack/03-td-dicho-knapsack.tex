%\documentclass[10pt]{article}\usepackage[correction,nu]{esial}
\documentclass[10pt]{article}\usepackage[nu]{esial}

\usepackage[utf8]{inputenc}
\usepackage{url}
\usepackage{amstext,amsmath,amsfonts}
\usepackage{fancyvrb}

\TOP\unA
\title{TDP3: Récursivité (dichotomie et sac à dos)}

\hypersetup{hidelinks}
\begin{document}
\maketitle

\begin{Reponse}
Cette semaine, on a un TD de (45mn de dichotomie/1h15 sac à dos), et un TP sur
le sac à dos. But du jeu: ne pas écrire le code du TP en TD, mais faire assez
d'exos d'échauffement sur le thème pour qu'ils écrivent le code seul sans
soucis ensuite. Si tout va bien, faut se décarcasser en TD et ne plus rien
avoir à faire en TP.
\end{Reponse}

\Exercice\textbf{Diviser pour régner: la dichotomie}

\begin{Reponse}
  L'objectif de cet exercice est d'appliquer les recettes de cuisine du cours
  pour (1) écrire une fonction récursive (voir le traitement appliqué aux tours
  de hanoi dans le cours) (2) dérécursiver une fonction récursive terminale.
  Evidement, sur un cas aussi simple, on peut faire directement au feeling,
  mais il faut insister sur la recette de cuisine histoire que tout le monde
  l'intègre.
\end{Reponse}

La dichotomie (du grec « couper en deux ») est un processus de recherche où
l'espace de recherche est réduit de moitié à chaque étape. 

Un exemple classique est le jeu de devinette où l'un des participants doit
deviner un nombre tiré au hasard entre 1 et 100. La méthode la plus efficace
consiste à effectuer une recherche dichotomique comme illustrée par cet
exemple:
\begin{itemize}
\item[--] Est-ce que le nombre est plus grand que 50? (100 divisé par 2)
\item[--] Oui
\item[--] Est-ce que le nombre est plus grand que 75? ((50 + 100) / 2)
\item[--] Non
\item[--] Est-ce que le nombre est plus grand que 63? ((50 + 75 + 1) / 2)
\item[--] Oui
\end{itemize}
On réitère ces questions jusqu'à trouver 65. Éliminer la moitié de l'espace de
recherche à chaque étape est la méthode la plus rapide en moyenne.

\begin{Question}
  Écrivez une fonction récursive cherchant l'indice d'un élément donné dans un
  tableau trié. La recherche sera dichotomique.

  \textsc{Données:}
  \begin{itemize}
  \item Un tableau trié de n éléments (\texttt{tab})
  \item Les bornes inférieure (\texttt{borne\_inf}) et
    supérieure (\texttt{borne\_sup}) du tableau
  \item L'élément cherché (\texttt{élément})
  \end{itemize}

  \textsc{Résultat:} l'indice où se trouve l'élément dans tab s'il y est, -1
  sinon.

\end{Question}
\begin{Reponse} 
  \textbf{Recette de cuisine:}
  \begin{itemize}
  \item Paramètre : l'intervale (et sa taille)
  \item Cas triviaux : si l'intervale est de taille 1 ou 0
  \item Comment résoudre le pb avec l'aide d'un bon génie : 
    \begin{itemize}
    \item Couper l'intervale en deux moitiés
    \item Regarder dans quelle moitié peut être l'élément cherché (le tableau
      est trié)
    \item Demander au génie de chercher pour de vrai dans cette moitié
    \end{itemize}
  \end{itemize}



  \textbf{Code:}
  Je pense qu'on peut faire plus simple que le code ci-après, mais pas
  sûr. Attention, si vous changez le test en \hbox{$min == max$}, ça merdoie
  grâce à cette saloperie de division entière contre-intuitive: ce cas
  (taille=0) n'arrive jamais si la taille initiale de l'intervale est impaire et
  que l'élément cherché est dans une case impaire (ou un plan du genre).

  Remarquez que dans ce code, je définis la fonction récursive \textbf{dans} la
  fonction publique. On peut tout à fait déclarer des fonctions dans les
  fonctions en scala, ca réduit leur visibilité comme on s'y attend. Dans ce
  cas, la fonction récursive est de la cuisine interne, et l'appellant n'a pas à
  savoir comment c'est implémenté dedans (ca lui évite de savoir initialiser les
  arguments récursifs).
  \newpage

  %\VerbatimInput[label=Version récursive]{src/dicho/DichoRec.java}
  \VerbatimInput[label=Version récursive]{dichoRec.scala}
  
\end{Reponse}

\Question Calculez la complexité de cette fonction.
\begin{Reponse}
  $O( log(n) )$ car on divise la quantité de chose à
  faire par 2 à chaque fois. Donc, en 1 étape, je gère 1 élément au max. En 2
  étapes, 2 fois plus. En 3 étapes, encore 2 fois plus. En N étapes, je gère
  $2^N$ éléments. Donc, pour gérer $p$ éléments, il faut $n$ étapes tel que
  $2^n>p$, ie, $n>log(p)$. CQFD.  
\end{Reponse}

\Question Montrez la terminaison de cette fonction.
\begin{Reponse}
  Quand on fait une fonction récursive, il est très important de montrer
  qu'elle s'arrête tout le temps, pour 2 raisons : (1) écrire une fonction
  récursive qui boucle à l'infini est facile et courant; (2) montrer ceci est
  relativement simple, c'est pas une preuve avancée.

  Une condition suffisante pour montrer la terminaison d'un algo recursif (mais
  pas forcément nécessaire), c'est de montrer qu'il y a une grandeur qui varie
  de façon strictement monotone, et qu'elle atteint à coup sûr les cas d'arrêt
  de la récursion. 

  \textbf{Ici:} la taille du tableau est divisé par 2 à chaque étape, et on
  s'arrête quand la taille est inférieure ou égale à 1.  Suite strictement
  monotone (attention aux divisions entières pour ca), effectivement
  convergeante vers le cas terminal, c'est bon.

  Si on avait marqué explicitement que la longueur entre min et max est 0 ou 1,
  on aurait pu se faire avoir avec des cas où max passe à gauche de
  min. Peut-être bien que cela aurait pu passer à travers, ie partir se
  promener chez les négatifs et donc partir à l'infini ($-\infty$). Mais cette
  simple précaution (écrire $i<=1$ quand on pense à $i\in[0,1]$) nous assure que ce
  ne sera pas le cas.
\end{Reponse}

\begin{Question} 
  Dérécursivez la fonction précédente.
\end{Question}
\begin{Reponse}
  \textbf{Cette question est optionnelle.}  Si votre groupe a posé trop de
  questions sur le début, passez à la suite. On refera de la dérécursivation, et
  il faut absolument défricher le knapsack en TD car c'est le sujet du TP.

  Il faut appliquer la recette de cuisine du cours. Tant que condition fausse,
  traitements du cas général. Ensuite, traitements du cas terminal.

  %\VerbatimInput[label=Version récursive]{src/dicho/DichoIter.java}
  \VerbatimInput[label=Version récursive]{dichoIter.scala}
\end{Reponse}

\Exercice\textbf{Présentation du problème du sac à dos}

L'objectif du prochain TP sera de réaliser un algorithme de recherche avec
retour arrière dans un graphe d'états. Nous allons maintenant nous familiariser
avec ce problème. 

Pour \textbf{éviter de boire la tasse sur machine}, il ne faut pas se lancer
dans l'implémentation sur machine tête baissée à l'exercice 4. Avant cela, il
faut absolument avoir fini et compris les exercices 2 et 3, qui préparent à
cette implémentation. Notez également qu'un code à compléter est fourni pour
l'exercice 4. Comme c'est la première fois que vous allez être amené à écrire du
scala hors de la PLM, il faut aussi prendre le temps de vérifier que votre
environnement de travail fonctionne. Mais avant tout, il s'agit de prendre les
choses dans l'ordre et réfléchir ensemble pour la suite de l'exercice 2.

\bigskip%
Le problème du sac à dos (ou \textit{knapsack problem}) est un problème
d'optimisation classique. L'objectif est de choisir autant d'objets que peut en
contenir un sac à dos (dont la capacité est limitée). Des problèmes similaires
apparaissent souvent en cryptographie, en mathématiques appliquées, en
combinatoire, en théorie de la complexité, \textit{etc.}  \medskip

\noindent\textsc{Problème:} \vspace{-.2\baselineskip}
\begin{quote}
  Étant donné un ensemble d'objets ayant chacun une valeur $v_i$ et un poids
  $p_i$, déterminer quels objets choisir pour maximiser la
  valeur de l'ensemble sans que le poids du total ne dépasse une borne $N$.\\
  (on pose dans un premier temps $\forall i, v_i=p_i$. Imaginez qu'il s'agit de
  lingots d'or de tailles différentes)
\end{quote}


\textsc{Données}:
\begin{itemize}
\item Le poids de chaque objet \texttt{i} (rangés dans un tableau
  \texttt{poids[0..n-1]})
\item La capacité du sac à dos \texttt{N}
\end{itemize}

\textsc{Résultat}:
\begin{itemize}
\item un tableau \texttt{pris[0..n-1]} de booléens indiquant pour chaque objet
  s'il est pris ou non
\end{itemize}

\medskip Le seul moyen connu\footnote{Ceci est du moins vrai dans la forme non
  simplifiée du problème et en utilisant des valeurs non entières.
  \ifcorrection{\color{red}Pour de vrai, dans le cas qu'on présente ici, on peut le
    claquer en temps polynomial par programmation linéaire. Mais c'est hors
    sujet...}{}} de résoudre ce problème est de tester différentes combinaisons
possibles d'objets, et de comparer leurs valeurs respectives.  Une première
approche serait d'effectuer une recherche exaustive d'absolument toutes les
remplissages possibles.

\Question Calculez le nombre de possibilités de sac à dos possible lors d'une
recherche exaustive.

\begin{Reponse}
  On devrait s'en sortir avec un $C_n^p$, mais on peut aussi constater qu'on a
  $n$ objets, et que chacun est soit pris, soit pas pris. Ce qui nous fait $n$
  caractères sur un alphabet à 2 lettres, soit $2^n$ possibilités.
\end{Reponse}

\medskip Une approche plus efficace consiste à mettre en œuvre un algorithme de
recherche avec retour arrière (lorsque la capacité du sac à dos est dépassée)
tel que nous le verrons en cours. Elle permet de couper court au plus tôt à
l'exploration d'une branche de l'arbre de décision. Par exemple, quand le sac
est déjà plein, rien ne sert de tenter d'ajouter encore des objets.

La suite de cet exercise vise à vous faire mener une réflexion préliminaire au
codage, que vous ferez dans l'exercise suivant, lors du prochain TP.

\Question Comment modéliser ce problème en Scala ? Par quel type de données allez
vous représenter un sac à dos donné ? 

\begin{Reponse}
  Dans le template qui leur sera donné, j'ai modélisé le sac à dos comme un
  tableau de boolean, chacun représentant si l'objet correspondant est pris ou
  non. Il faut donc parvenir à les faire converger vers cette solution ;) 
\end{Reponse}

% \Question De combien de classes Java avez vous besoin? Dessinez le
% diagramme de classes.

% \begin{Reponse}
%   La première tentation de certain est de répondre ``on a pas fait d'UML à
%   l'ESIAL''. Ceux qui disent ça ont déjà fait de l'UML ailleurs vu qu'ils font
%   le rapprochement avec ``diagramme de classes''; sinon, pour détendre
%   l'atmosphere, j'ai jamais fait d'UML, perso, ni à l'esial ni ailleurs, c'est
%   pas ça qui empêche de réfléchir aux classes à écrire et aux fonctions
%   qu'elles doivent offrir...

%   Sinon, il faut couper immédiatement court à l'idée de faire un objet Java
%   pour chaque objet qu'on va mettre (ou non) dans  le sac à dos. Les deux ont
%   le même nom, mais les objets du problème sont passifs là où un objet Java est
%   actif (quelles seraient les méthodes à écrire dans la classe
%   ``KnapsackObjet''?).

%   \textbf{Faut bien les faire ramer avant de leur donner le schéma suivant.}
%   Mais faut pas les laisser ramer seuls, faut guider leurs recherches en
%   posant des questions et en écrivant peu à peu interactivement. L'autre piège
%   est qu'il faut subtilement guider cette ``interactivité'' pour s'assurer
%   que le groupe retrouve ce schéma là, puisque c'est celui utilisé dans les 
%   templates fournis lors du TP. Mener le groupe à un autre design serait
%   alors troublant.

%   \includegraphics{fig/diagramme-classe.fig}
% \end{Reponse}

\Question Écrivez les méthodes \texttt{mettreDansSac()}, \texttt{retireDuSac()} et
\texttt{estPris()} qui simplifient l'usage de votre structure de données.

\begin{Reponse}
  Ces méthodes sont triviales, et cette question a pour objectif principal que
  tout le monde comprend bien la modélisation utilisée.
  \begin{Verbatim}
/** indique dans le paramètre que l'objet spécifié est maintenant pris
    (cette fonction est juste là pour se simplifier la vie) */
def mettreDansSac(objets:Array[Boolean], obj:Int) {
    if (objets(obj)) 
        println("L'objet "+obj+" est déjà pris; ignore la requete");
    objets(obj) = true
}
/** indique dans le paramètre que l'objet spécifié est maintenant posé */
def retireDuSac(objets:Array[Boolean], obj:Int) {
    if (!objets(obj)) 
        println("L'objet "+obj+" est déjà posé; ignore la requete");
     objets(obj) = false
}
/** ben oui, c'est juste un acces au tableau, quoi */
def estPris(objets:Array[Boolean], obj:Int) = objets(obj)
  \end{Verbatim}
\end{Reponse}

\Question Écrivez une fonction \texttt{valeurTotale}, prennant un sac à dos et
un tableau d'entiers \texttt{poids} en paramètre (\texttt{poids} indique le
poids de chaque objet possible du sac à dos) et renvoyant le poids total de tous
les objets sélectionnés pour le sac à dos.

\begin{Reponse}
  \textbf{Il faut bien les laisser ramer avant de leur filer ce code, sinon ils
    vont s'ennuyer en TP.} Si vous donnez tout le code au tableau en TD, ils
  vont râler (à juste titre) en réunion pédagogique que le travail en TD est de
  recopier le tableau sans réfléchir, puis que le travail en TP est de taper le
  TD sur l'ordi sans réfléchir.

  \textbf{Je ne serais pas choqué si vous ne corrigiez pas cette méthode au
    tableau}, au fond. Donner l'idée générale suffira. Par exemple: ``Je
  remarque que les deux tableaux doivent être de même taille pour que ça
  marche. Pour trouver la valeur, j'initialise une variable \texttt{total }à 0,
  puis pour tous les indices possibles, si l'objet est pris, j'ajoute son poids
  au total en cours de calcul. A la fin, la valeur renvoyée est
  \texttt{total}''.
\begin{Verbatim}
/** Calcule le gain d'un sac à dos donné */
def valeurTotale(objets:Array[Boolean], poids:Array[Int]):Int = {
  var total = 0
  for (i <- 0 to objets.length -1)
    if (objets(i)) 
      total += poids(i)
  return total
}  
\end{Verbatim}
\end{Reponse}

\Exercice\textbf{Résolution récursive du problème du sac à dos}

\Question Dessinez l'arbre d'appel que votre fonction doit parcourir lorsqu'il
y a quatre objets de tailles respectives $\{ 5, 4, 3, 2 \}$ pour une capacité
de 15.

\begin{Reponse}
  Dans le schema ci-dessous, la moitié gauche, sous ``5'' est les cas où on
  prend le premier objet, de taille 5. La moitié droite, sous ``5 barré'' est
  les cas où on prend pas 5.

  \includegraphics[width=\linewidth]{fig/arbre-appels.fig}  

  Ouaip, il faut explorer tout l'arbre car le sac est trop grand, tout tient
  dedans -- même pas drôle.
\end{Reponse}

\Question Même question avec un sac de capacité 8.

\begin{Reponse}
  Là, on peut  couper pas mal de branches. Par exemple, on n'a pas besoin d'aller au
  delà de 5,4 car le sac éclate déjà avec seulement 5 et 4. Ça ne risque donc
  pas d'aller mieux en ajoutant plus de choses. Donc pas besoin d'explorer le
  sous-arbre gauche.

  Attention cependant, nous n'avons pas vu la notion de backtracking en
  cours. Mais c'est pas grave, car ce problème n'est pas vraiment un
  backtracking vu qu'il n'y a pas de boucle.
\end{Reponse}

\medskip Réfléchissez à la structure de votre programme. Il s'agit d'une
récursion, et il vous faut donc choisir un paramètre portant la
récursion. Appliquez le même genre de réflexion que celui mené en cours à
propos des tours de Hanoï. Il est probable que vous ayez à introduire une
fonction récursive dotée de plus de paramètres que la méthode
\texttt{cherche()}.

\Question Quel sera le prototype de la fonction d'aide? Quel sera le prototype
de la fonction récursive?

\begin{Reponse}
  Le mieux est de faire un prototype qui soit exactement celui de l'énoncé pour
  la fonction d'aide. Une ruse est de définir la fonction récursive
  \textbf{dans} la fonction d'aide. Comme ça, on est sûrs que personne ne va
  l'appeller en initialisant les paramètres n'importe comment, et en plus, ca
  lui permet de voir les paramètres de \texttt{cherche()}
  \begin{Verbatim}
/** La fonction publique, pour chercher la meilleure solution */
def cherche(poids:Array[Int] , capacite:Int) {

  /* Appel à chercheRec ici, et affichage du résultat */
  
  def chercheRec(profondeur:Int, courante:Array[Boolean]) {
    /* Code récursif ici */
  }
}    
  \end{Verbatim}
\end{Reponse}

\medskip Dans le même temps, l'objectif de votre méthode récursive est de
trouver le meilleur nœud de l'arbre visité. D'une certaine façon, c'est assez
similaire à une recherche de l'élément minimal dans un tableau: on parcours tous
les éléments, et pour chacun, s'il est préférable au meilleur candidat connu
jusque là, on décrète qu'il s'agit de notre nouveau meilleur candidat. Mais
contrairement à l'algorithme d'une recherche d'élément minimum, les valeurs ne
sont pas de simples entiers. 

\Question Écrivez une fonction qui recopie une structure représentant un sac à
dos dans une autre. On l'utilisera pour sauvegarder le nouveau meilleur
candidat.

\begin{Reponse}
  \begin{Verbatim}
def dupplique(src:Array[Boolean], dst:Array[Boolean]) {
  for (i <- 0 to src.length-1) 
    dst(i) = src(i)
}    
  \end{Verbatim}
\end{Reponse}

\Question Explicitez en français l'algorithme à écrire. Le fonctionnement en
général (en vous appuyant sur l'arbre d'appels dessiné à la question 3), puis
l'idée pour chaque étage de la récursion.

\begin{Reponse}
  Ce qui semble mal passer, c'est que ce n'est pas un if/then/else, mais une
  séquence. Je prend l'objet, je descend [à gauche], je le pose, je descend [à
  droite]. À chaque étage, c'est très similaire une fois qu'on a vu que c'est
  à  l'objet N qu'on s'intéresse à l'étage N.

  Tâchez de ne pas prononcer le mot backtracking car (1) ils ne savent pas ce
  que c'est (2) ce n'est pas vraiment un backtracking comme on verra en cours vu
  qu'on a pas de boucle au sein de la récursion. Au lieu de parcourir l'ensemble
  \{pris, pas pris\} avec une vraie boucle, on teste les deux cas explicitement.

  Il faut qu'ils comprennent l'idée de couper des branches dans une recherche
  exhaustive, car on recommence la semaine prochaine...
\end{Reponse}

\Exercice\textbf{Implémentation d'une solution au problème du sac à dos}

Récupérez les fichiers \textbf{knapsack.jar} et \textbf{knapsack.scala} depuis
la page du cours\footnote{Page du cours:
  \url{http://www.loria.fr/~quinson/Teaching/TOP}} ou depuis le
dépôt\footnote{Dépot: \url{/home/depot/1A/TOP/knapsack} sur les serveurs de
  l'école.}.

\Question Expérimentez avec le fichier \textbf{knapsack.jar}. Il s'agit d'une
version compilée exécutable de la solution. Exécutez-le en lui passant diverses
instances du problème en paramètre (en ligne de commande) afin de mieux
comprendre comment fonctionne l'algorithme que vous devez écrire.

\begin{Reponse}
  Pas de correction, faut juste qu'ils fassent mumuse avec le binaire fourni
  jusqu'à être à l'aise.
\end{Reponse}

\Question Le fichier \textbf{knapsack.scala} également fourni est un template de
la solution, c'est-à-dire une sorte de texte à trou de la solution que vous
devez maintenant compléter. Testez votre travail sur plusieurs instances du problème.

\begin{Reponse}
  \noindent\textbf{Ce sont les corrections du TP suivant, alors c'est pas une
    bonne idée de tout leur donner : faut qu'ils cherchent, un peu,
    aussi.}
    
  \VerbatimInput{scala/knapsack-solution.scala}
\end{Reponse}


\Question Combien d'appels récursifs effectue votre code?

\begin{Reponse}
  Certains étudiants  vérifient la validité de leur travail en comptant le
  nombre d'appels récursifs effectués. Bonne idée. Le bon résultat pour
  l'instance \{5,4,3,3\} (capacité 10) est 15. Pas
  le nombre de feuilles de l'arbre, mais le nombre de nœuds internes...
\end{Reponse}

\Question Testez votre travail pour des instances plus grandes du problème et
observez les variations du temps d'exécution lorsque la taille du problème
augmente.

\begin{Reponse}
  Oui, c'est exponentiel. Un peu mieux que $2^n$, mais ça reste très douloureux
  très vite.
\end{Reponse}

\Question Généralisez votre solution pour résoudre des problèmes où la valeur
d'un objet est décorrélée de son poids (on ne suppose donc plus que $v_i=p_i$).
Il s'agit maintenant de maximiser la valeur du contenu du sac en respectant les
contraintes de poids. Vous serez pour cela amené à modifier toutes les fonctions
écrites. 

\end{document}
%%% Local Variables:
%%% coding: utf-8
