\documentclass[10pt]{article}\usepackage[correction,nu]{esial}

\usepackage[utf8]{inputenc}
\usepackage{url}
\usepackage{amstext,amsmath,amsfonts}
\usepackage{fancyvrb}

\TOP
\title{TD2 : Récursivité et chaînes récursives}

\begin{document}
\maketitle

\noindent\begin{minipage}{.57\linewidth}
\Exercice Code mystère.
\Question Calculez les valeurs renvoyées par la fonction f pour n variant entre
1 et 5. 
\Question Quelle est la fonction mathématique vue en cours que f() calcule?
\Question Quelle est la complexité algorithmique du calcul ? 
\end{minipage}\hfill\begin{minipage}{.4\linewidth}
\begin{Verbatim}
def g(n:Int, a:Int, b:Int):Int ={
  if (n == 0) {
    return a;
  } else {
    return g(n-1,b,a+b);
  }
}
def f(n:Int):Int ={
  return g(n,0,1);
}  
\end{Verbatim}  
\end{minipage}\smallskip

Notez que tout le
travail est fait par la fonction g, et la fonction f ne sert qu'à donner une
valeur initiale aux arguments a et b, qui servent d'accumulateur. Il s'agit là
d'une technique assez classique en récursivité.

\begin{Reponse}
  C'est fibonnacci, bien sûr. Et c'est bô car c'est du récursif avec une
  complexité algorithmique linéaire. C'est l'occasion de dire que si la forme
  classique de fibo est aussi nulle en perf, c'est pas tant à cause de la
  récursivité, mais plutot à cause de la façon dont elle est écrite, c'est
  tout. 

  Remarquez aussi que les algos classiques itératifs sont linéaires en temps,
  mais également linéaires en espace vu qu'ils font un gros tableau des
  résultats temporaires déjà rencontrés. Ce n'est pas le cas de cette approche,
  qui est bien sûr aussi applicable en itératif.

  Le message est ``la récursion n'est ni plus ni moins efficace que
  l'itératif'' et ``Cette approche est la plus efficace que je connaisse''
  (meme si je m'amuserais pas à avancer qu'elle est optimale, il est possible
  qu'une fonction en temps constant existe pour calculer fibo, après tout).
\end{Reponse}

\Exercice Soit le type \texttt{Chaine} muni des opérations suivantes :
$$\left\{
\begin{array}{l}
  chvide:  \emptyset\mapsto Chaine\\
  premier: Chaine\mapsto Caract\grave{e}re \;\;\text{(défini ssi la chaîne 
    n'est pas vide)}\\
  reste:   Chaine\mapsto Chaine \text{\hspace{11mm}(défini ssi la chaîne 
    n'est pas vide)}\\
  adj:     Chaine\times Caract\grave{e}re \mapsto Chaine
\end{array}\right.
$$

\noindent Écrire les fonctions suivantes. Vous préciserez les préconditions
nécessaires.

\begin{Reponse}
  \textbf{Comment lancer la séance:} ``les ptits malins qui savent
  déjà tout, vous avez 20 questions devant vous, pas la peine de nous attendre,
  on va prendre le temps de comprendre. Avancez. Indication: toutes les
  questions admettent des réponses linéaires en temps, sauf 2. L'une est un peu
  pire que O(n), l'autre est meilleure. Bonne chance''. Et ensuite, on prend
  les vrais débutants par la main.

  \textbf{A propos des preuves}
  Je suis désolé, vu que j'ai 2 séances cette semaine et pis plus rien pendant
  un grand moment, et pleins de TD/TP entre temps, j'ai pas eu le temps de
  parler de preuve de programme, encore. Du coup, il faut leur expliquer avec
  les mains ce qu'on veut faire à propos de la terminaison. Pourquoi c'est
  important et comment on le montre... Laissez tomber la correction, sauf pour
  dire ``on voit bien'' et déplier des exemples quand le premier argument
  suffit pas.

  \textbf{Ce qui est important de faire:}
  \begin{itemize}
  \item Les questions de base, avec récursivité simple. Il faut appliquer à la
    lettre la recette de cuisine du cours:
    \begin{itemize}
    \item identifier sur quoi porte la récursion (ici, c'est tjs la longueur de
      la chaine)
    \item Identifier et résoudre les cas triviaux (ici, c'est souvent quand la
      chaine est vide, plus de temps en temps quand le premier est ce qu'on
      cherche)
    \item Faire le cas général, ie faire le pb pour la chaine complete en
      supposant que ``quelqu'un'' sait faire quand la chaine est plus courte.
    \end{itemize}
  \item Des questions où y'a une remontée récursive plus intelligente, ie, tous
    ceux qui font des \textit{adj()} avec la récursion sur l'un de ses
    arguments.
  \item Des questions avec précondition. On se prend pas la tete, on l'écrit
    juste pour signifier ``si quelqu'un est assez bête pour appeller la
    fonction dans un cas où c'est pas respecté, ca va mal se passer''. Pas
    besoin de vérifier explicitement la longueur de la chaine, par exemple. On
    indique juste ``Précondition: la chaine est assez longue''.
  \item Des questions où y'a besoin d'un helper pour aller plus vite. Par
    exemple \textit{nderniers} (dans sa 2ieme version) ou
    \textit{retourne}. \textbf{C'est important.}
  \item Dans un monde parfait, il faut tout faire jusqu'à concat.
  \end{itemize}

  \textbf{Ce qu'il est important de dire:}
  \begin{itemize}
  \item Insister sur la terminaison (meme si on le fait avec les mains). Ca
    termine car la longueur de la chaine est strictement décroissante et que
    j'ai un cas terminal pour lgr=0. Il faut aussi dire que décroissante + cas
    terminal en 0 est pas assez : si on décroit de 2 en 2 en partant d'un
    impaire, on ``passe à travers''. Mais c'est pas le cas ici. 

    Faire sentir tout ca, meme si on démontre rien.
  \item Le coût algorithmique de chaque fonction. Souvent $\Theta(n)$, parfois
    $O(n)$, parfois autre.
  \item Insister sur l'intérêt des fonctions Helper, et comment on les
    construit : les arguments supplémentaires sont des accumulateurs dans
    lesquels le résultat se construit peu à peu (exemple de retourne ou
    concat). Ou alors dans lesquels une donnée précalculée est stockée (exemple
    de Nderniers).
  \end{itemize}
\end{Reponse}

%%%%%%%% longueur %%%%%%%%%%%%%%%%%%%%%%%%%%%%%%%%%%%%
\begin{Question}
  $longueur: \left\{
    \begin{array}{l}
      Chaine\mapsto \mathbb{N}\\
      \text{retourne le nombre de lettres composant la chaîne}
    \end{array}\right.$
\end{Question}
\begin{Reponse}
  \begin{Verbatim}[label=longueur(ch)]
si ch = chvide alors 0
               sinon 1 + longueur(reste(ch))    
  \end{Verbatim}
  \begin{description}
    \item
    \item[Idée pour trouver comment faire] Imaginez que vous voulez savoir
      combien de camions sont devant vous sur cette petite route de
      montagne. Vous les voyez pas tous.
      \begin{itemize}
      \item Seule solution, vous en doublez 1, et vous savez qu'au total, il y
        en avait 1+ ce qui vous reste à doubler.
      \item  Vous en doubler un autre, et au total, il y en avait 1+1+ce qui
        vous reste à doubler.
      \item Le jour ou vous avez plus de camion devant vous, il vous en reste 0
        à doubler, et au total, il y en avait 1+1+1+1+...+1+0.
      \end{itemize}
  \item[Terminaison:] La longueur est strictement décroissante et on s'arrête
    quand la chaîne est vide.
  \item[Correction:] On se contente de déplier les appels au tableau
    aujourd'hui. Mais il faut le faire pour qu'ils comprennent, et il faut le
    faire à chaque fois, pas que celui là. \textbf{Pas cette année, j'ai pas
      présenté les preuves de prog encore. L'an prochain ca sera dans le bon
      ordre.} 
  \item[Complexité:] On regarde chaque lettre, on a donc O(n) appels
    récursifs. A chaque appel, on n'appelle que des fonctions de base, pas
    chères. Donc une étape est en O(1). Résultat: $O(n)\times O(1)=O(n)$
  \end{description}
\end{Reponse}

%%%%%%%% est_membre %%%%%%%%%%%%%%%%%%%%%%%%%%%%%%%%%%%%
\begin{Question}
  $est\_membre: \left\{
    \begin{array}{l}
      Chaine\times caract\grave{e}re\mapsto bool\acute{e}en\\
      \text{retourne VRAI ssi le caractère fait partie de la chaîne}
    \end{array}\right.$  
\end{Question}
\begin{Reponse}
  \begin{Verbatim}[label=est\_membre(ch\quotesinglbase c)]
si ch = chvide alors FAUX
               sinon si premier(ch) = c alors VRAI
                                        sinon est_membre(reste(ch),c)    
  \end{Verbatim}
  \begin{description}
  \item[Est ce que cette fonction traite correctement le cas où le caractère
    n'est pas dedans] Appliquez le meme algo à la recherche d'une peau rouge
    dans un oignon. Je regarde la peau extérieure, elle est pas rouge, je
    l'enlève et recommence. Je recommence sur toutes les peaux (toutes jaunes)
    jusqu'à la toute dernière. Je l'enlève aussi car elle est jaune. Je me
    retrouve avec l'oignon vide entre les mains, j'ai donc l'assurance
    qu'aucune peau n'était rouge dans mon oignon.
  \item[Terminaison et Complexité:] comme avant. Simplement, chaine vide n'est
    pas le seul cas terminal, mais ca ne gène pas la terminaison. On pourrait
    chercher à faire une étude plus précise de la complexité avec meilleur des
    cas et pire des cas, mais ce n'est pas la peine. En moyenne, cet algo est
    linéaire. Faut juste leur faire sentir la complexité et laisser au module
    "Mat Num" le plaisir de faire des maths.
  \item[Complexité meilleur des cas] c'est si la chaine est vide ou qu'on
    cherche 't' dans 'toto'. $O(1)$
  \item[Complexité pire des cas] Je cherche 'e' dans 'toto', je dois parcourir
    toute la chaine. $O(n)$
  \item[Complexité cas moyen] Ben on peut pas répondre avec si peu de
    données. Ceux qui répondent [$\frac{n}{2}$] supposent l'équiprobabilité des
    lettres, c'est une hypothèse forte que l'on a pas. Imaginez chercher le ß
    (on le z) dans la langue française, par rapport au 'e'.
  \end{description}
\end{Reponse}

%%%%%%%% occurence %%%%%%%%%%%%%%%%%%%%%%%%%%%%%%%%%%%%
\begin{Question}
  $occurence: \left\{
    \begin{array}{l}
      Chaine\times caract\grave{e}re\mapsto \mathbb{N}\\
      \text{retourne le nombre d'occurences du caractère dans la chaine}
    \end{array}\right.$  
\end{Question}
\begin{Reponse}
  \begin{Verbatim}[label=occurences(ch\quotesinglbase c)]
si ch = chvide alors 0
               sinon si premier(ch) = c alors 1 + occurence(reste(ch),c)
                                        sinon     occurence(reste(ch),c)    
  \end{Verbatim}
  \begin{description}
  \item[Terminaison et Complexité:] comme avant: $O(n)$. 
  \end{description}
\end{Reponse}

%%%%%%%% tous_differents %%%%%%%%%%%%%%%%%%%%%%%%%%%%%%%%%%%%
\begin{Question}
  $tous\_differents: \left\{
    \begin{array}{l}
      Chaine\mapsto bool\acute{e}en\\
      \text{retourne VRAI ssi tous les membres de la chaine sont différents}
    \end{array}\right.$  
\end{Question}
\begin{Reponse}
  \begin{Verbatim}[label=tous\_differents(ch)]
si ch = chvide alors VRAI
               sinon si est_membre(suite(ch), premier(ch)) alors FAUX
                                                     sinon tous_differents(suite(ch))    
  \end{Verbatim}
  \begin{description}
  \item[Terminaison:] comme avant.
  \item[Complexité:] On fait toujours $O(n)$ appels récursifs, mais ce coup ci,
    chacun fait appel à est\_membre, qui est elle même en $O(n)$. Donc $C =
    O(n)\times O(n) = O(n^2)$

    On peut se poser la question de l'optimalité. Est ce que c'est comme ca que
    vous vérifiez que toutes les cartes d'un paquet sont différente ?? Non,
    bien sur. Le plus simple à la main, c'est de trier la pile dans un ordre
    donné, puis de faire un seul parcours en comparant chaque carte à la
    suivante (un peu comme la fonction croissante, donnée plus bas).

    Étant donné qu'il existe des algos de tris en $O(n\times log(n))$, on a
    $$C = C_{pretraitement}+C_{fonction recursive}= O(n\times log(n)) + O(n) =
    O(n\times log(n))$$. Ce qui est bien mieux que $O(n^2)$ quand n est grand.

    Notons cependant que la complexité dans le meilleur des cas passe de $O(1)$
    (quand la chaîne commence par 'aa', l'algo $O(n^2)$ répond immédiatement) à
    $O(n\times log(n)$... sauf si on a fait son tri avec attention.
  \end{description}
\end{Reponse}

%%%%%%%% supprime %%%%%%%%%%%%%%%%%%%%%%%%%%%%%%%%%%%%
\begin{Question}
  $supprime: \left\{
    \begin{array}{l}
      Chaine\times caract\grave{e}re\mapsto Chaine\\
      \text{retourne la chaine privée de la première occurence du caractère.}
    \end{array}\right.$  

  Si le caractère ne fait pas partie de la chaîne, celle-ci est inchangée.
\end{Question}
\begin{Reponse}
  \begin{Verbatim}[label=supprime(ch\quotesinglbase c)]
si ch = chvide alors ch
               sinon si premier(ch) = c alors reste(ch)
                                        sinon adj(premier(ch), supprime(suite(ch),c))
  \end{Verbatim}
  \begin{description}
  \item[Terminaison et Complexité:] comme avant: $O(n)$. 
  \end{description}
\end{Reponse}

%%%%%%%% deuxieme %%%%%%%%%%%%%%%%%%%%%%%%%%%%%%%%%%%%
\begin{Question}
  $deuxieme: \left\{
    \begin{array}{l}
      Chaine\mapsto caract\grave{e}re\\
      \text{retourne le deuxième caractère de la chaîne}
    \end{array}\right.$  
\end{Question}
\begin{Reponse}
  \begin{Verbatim}[label=deuxieme(ch)]
PRECONDITION: ch != vide et suite(ch) != vide
premier(suite(ch))
  \end{Verbatim}
  \begin{description}
  \item[Terminaison:] C'est un appel direct, sans récursion. Mais c'est
    l'occasion de réintroduire les préconditions.
  \item[Complexité:] O(1)
  \end{description}
\end{Reponse}

%%%%%%%% dernier %%%%%%%%%%%%%%%%%%%%%%%%%%%%%%%%%%%%
\begin{Question}
  $dernier: \left\{
    \begin{array}{l}
      Chaine\mapsto caract\grave{e}re\\
      \text{retourne le dernier caractère de la chaîne}
    \end{array}\right.$  
\end{Question}
\begin{Reponse}
  \begin{Verbatim}[label=dernier(ch)]
PRECONDITION: ch != vide
si suite(ch) = vide alors premier(ch)
                    sinon dernier(suite(ch))    
  \end{Verbatim}
  \begin{description}
  \item[Terminaison et Complexité:] comme avant: $O(n)$. 
  \end{description}
\end{Reponse}

%%%%%%%% saufdernier %%%%%%%%%%%%%%%%%%%%%%%%%%%%%%%%%%%%
\begin{Question}
  $saufdernier: \left\{
    \begin{array}{l}
      Chaine\mapsto Chaine\\
      \text{retourne la chaine privée de son dernier caractère}
    \end{array}\right.$  
\end{Question}
\begin{Reponse}
  \begin{Verbatim}[label=saufdernier(ch)]
PRECONDITION: ch != chvide
si suite(ch) = chvide alors chvide
                      sinon adj(premier(ch), saufdernier(suite(ch)))    
  \end{Verbatim}
  \begin{description}
  \item[Terminaison et Complexité:] comme avant: $O(n)$. 
  \end{description}
\end{Reponse}

%%%%%%%% nieme %%%%%%%%%%%%%%%%%%%%%%%%%%%%%%%%%%%%
\begin{Question}
  $nieme: \left\{
    \begin{array}{l}
      Chaine\times\mathbb{N}\mapsto caract\grave{e}re\\
      \text{retourne le nieme caractère de la chaîne}
    \end{array}\right.$  
\end{Question}
\begin{Reponse}
  \begin{Verbatim}[label=nieme(ch\quotesinglbase n)]
PRECONDITION: ch != vide
si n = 0 alors alors premier(ch)
               sinon nieme(suite(ch), n-1)    
  \end{Verbatim}
  \begin{description}
  \item[Terminaison et Complexité:] comme avant: $O(n)$. 
  \end{description}
\end{Reponse}

%%%%%%%% npremiers %%%%%%%%%%%%%%%%%%%%%%%%%%%%%%%%%%%%
\begin{Question}
  $npremiers: \left\{
    \begin{array}{l}
      Chaine\times\mathbb{N}\mapsto Chaine\\
      \text{retourne les n premiers caractères de la chaîne}
    \end{array}\right.$  
\end{Question}
\begin{Reponse}
  \begin{Verbatim}[label=npremiers(ch\quotesinglbase n)]
PRECONDITION: n>=longueur(ch)
si n = 0 alors chvide
         sinon adj(premier(ch), npremiers(n-1,suite(ch)))    
  \end{Verbatim}
  \begin{description}
  \item[Terminaison et Complexité:] comme avant: $O(n)$.
  \item[Correction:] C'est un bon exemple pour faire une preuve de correction:
    \begin{itemize}
    \item Précondition à l'étape $n$ entraine (récursivement) la précondition
      pour les étapes suivantes avec des $n$ plus petits
    \item Le traitement dans le cas terminal (pour $n=0$) assure la
      post-condition
    \item Le traitement lors de la remontée assure la post-condition
    \end{itemize}
  \end{description}
\end{Reponse}

%%%%%%%% nderniers %%%%%%%%%%%%%%%%%%%%%%%%%%%%%%%%%%%%
\begin{Question}
  $nderniers: \left\{
    \begin{array}{l}
      Chaine\times\mathbb{N}\mapsto Chaine\\
      \text{retourne les n derniers caractères de la chaîne}
    \end{array}\right.$  
\end{Question}
\begin{Reponse}
  Plusieurs variantes sont possibles:
  \begin{Verbatim}[label=nderniers (ch\quotesinglbase n) -- version naive]
ndernier(ch,n) = si lgr(ch)=n alors ch
                 sinon ndernier(reste(ch), n)    
  \end{Verbatim}
  \begin{description}
  \item[Complexité:] On a $O(n)$ appels récurssifs, mais chacun fait un appel à
    longueur, qui est elle même en $O(n)$. Donc, $C=0(n)\times
    O(n)=O(n^2)$. 
  \end{description}

  \begin{Verbatim}[label=nderniers (ch\quotesinglbase n) -- avec retourne]
ndernier(ch,n) = retourne(npremiers(retourne(ch), n))
  \end{Verbatim}
  \begin{description}
  \item[Complexité:] On ajoute les complexités respectives de chaque appel. $C
    = O(n) + O(n) + O(n) = O(n)$, (car $C_{retourne}=O(n)$) ce qui est
    mieux. Mais retourne n'est pas encore défini. Alors on en fait une
    troisième qui est l'occasion d'introduire les fonctions helpers.
  \end{description}

  \begin{Verbatim}[label=nderniers (ch\quotesinglbase n) -- avec fonction d'aide]
ndernier(ch,n) = supp_n_premiers(ch,lgr(ch)-n)

supp_n_premiers(ch, k) =
   si k=0 alors ch sinon supp_n_premiers(reste(ch), k-1)
  \end{Verbatim}

  L'idée est donc de calculer une bonne fois pour toute combien de caractères
  il faut retirer, puis de le faire ensuite sans réflechir au lieu de (comme
  dans la première) regarder apres chaque retrait si on en a enlevé assez. Ca
  permet de tomber la complexité en $O(n)$.
\end{Reponse}

%%%%%%%% retourne %%%%%%%%%%%%%%%%%%%%%%%%%%%%%%%%%%%%
\begin{Question}
  $retourne: \left\{
    \begin{array}{l}
      Chaine\mapsto Chaine\\
      \text{retourne la chaine lue en sens inverse}
    \end{array}\right.$  
\end{Question}
\begin{Reponse}
  Cette fonction est très importante. Si vous manquez de temps, faites sauter
  d'autres fonction pour parvenir à faire celle là car on en a très besoin dans
  le TP2. Là encore, il y a plusieurs solutions.
  \begin{Verbatim}[label=retourne(ch) -- version bourinne]
si ch = chvide alors chvide
               sinon adj(dernier(ch), retourne1(saufdernier(ch)))
  \end{Verbatim}
  \begin{description}
  \item[Complexité:] $O(n)$ appels, et $O(n)$ chaque à cause de saufdernier et
    dernier. $O(n^2)$, donc.
  \end{description}
  
  \textbf{Comment leur faire trouver mieux:} Demandez leur de réfléchir à
  comment ils inversent l'ordre d'une pile de cartes : on prend une pile
  supplémentaire, on passe le premier de la pile de départ sur l'autre, et on
  recommence avec la deuxieme de la pile de départ. Si ca aide pas, faut
  détailler un exemple:
  A trier ABC $\leadsto$ \begin{tabular}{|l l|}\hline
    ABC&$\emptyset$\\
    BC&A\\
    C&BA\\
    $\emptyset$&CBA\\\hline
  \end{tabular}$\leadsto$ résulat = CBA

  \begin{Verbatim}[label=retourne(ch) -- avec helper]
retourne2(ch):
  retourne2_helper(ch,chvide)

retourne2_helper(ch_todo,ch_done):
  si ch_todo = chvide alors ch_done
                      sinon retourne2_helper(suite(ch_todo),
                                             adj(premier(ch_todo), ch_done))
  \end{Verbatim}
  Comme souvent avec les fonctions helpers, on construit dans un argument
  supplémentaire le résultat final. Donc, on prend le travail qu'on aurait fait
  pendant la remontée, et on le fait dans la descente sur cet
  accumulateur. C'est important car ca change la fonction en récursive
  terminale (même s'ils n'ont pas encore vu ce que c'est à ce moment du cours).

  Ce qui nous interresse ici, c'est que la complexité passe en $O(n)$. Il est
  très important de déplier le retournement d'une chaîne d'exemple avec cette
  méthode. 

\end{Reponse}

%%%%%%%% concat %%%%%%%%%%%%%%%%%%%%%%%%%%%%%%%%%%%%
\begin{Question}
  $concat: \left\{
    \begin{array}{l}
      Chaine\times Chaine\mapsto Chaine\\
      \text{retourne les deux chaines concaténées}
    \end{array}\right.$  
\end{Question}
\begin{Reponse}
  \begin{Verbatim}[label=version brutale: $O(n^2)$]
concat1(ch1,ch2):
  si ch1 = chvide alors ch2
                  sinon concat1(saufdernier(ch1),
                                adj(dernier(ch1), ch2))
  \end{Verbatim}

<<<<<<< HEAD:TD/02-td-recursivite/02-td-recursivite.tex
  Pour aller plus vite, il faut laisser ch1 à l'envers le temps de travailler
  au lieu de passer son temps à aller à la fin des chaines. On passe de
=======
  Pour aller plus vite, il faut mettre ch1 à l'envers une bonne fois pour toute
  au lieu d'aller piocher le dernier à tout bout de champ. On passe de
>>>>>>> 782d82c154b08feb3830f98a7940e2cc8a5e24ce:TD/02-td-recursivite/02-td-recursivite.tex
  $O(n^2)$ à $O(n)$, tout de même. Encore une fois, un exemple donné avant les
  aide à trouver tous seuls.

  \begin{tabular}{|l l l|}\hline
    ABC&DEF& en donnée\\
    CBA&DEF& on inverse ch1 avant d'appeller helper\\
    BA&CDEF& récursion dans helper\\
    A&BCDEF& récursion dans helper\\
    $\emptyset$&ABCFED& Cas terminal de la récursion dans helper\\\hline
  \end{tabular}$\leadsto$ résultat = ABCDEF

  \begin{Verbatim}[label=version avec helper: $O(n)$]
concat2_helper(ch1,ch2): (un peu mieux)
  si ch1 = chvide alors ch2
                  sinon concat2_helper(suite(ch1),
                                       adj(premier(ch1),ch2))

concat2(ch1,ch2):
  concat2_helper(retourne(ch1),ch2)    
  \end{Verbatim}
\end{Reponse}

%%%%%%%% min_ch %%%%%%%%%%%%%%%%%%%%%%%%%%%%%%%%%%%%
\begin{Question}
  $min\_ch: \left\{
    \begin{array}{l}
      Chaine\mapsto caract\grave{e}re\\
      \text{retourne le caractère le plus petit de la chaîne}
    \end{array}\right.$  
  
  \smallskip
  On considère l'ordre lexicographique, et on suppose l'existance d'une
  fonction min(a,b).
\end{Question}
\begin{Reponse}
  \begin{Verbatim}[label=min\_ch(ch)]
PRECONDITION: ch != chvide
si suite(ch) = chvide alors premier(ch)
                      sinon min(premier(ch), min_ch(suite(ch)))
  \end{Verbatim}

  \begin{description}
  \item[Terminaison et Complexité:] comme avant: $O(n)$. 
  \end{description}
\end{Reponse}

%%%%%%%% croissante %%%%%%%%%%%%%%%%%%%%%%%%%%%%%%%%%%%%
\begin{Question}
  $croissante: \left\{
    \begin{array}{l}
      Chaine\mapsto bool\acute{e}en\\
      \text{retourne si la chaine est croissante (dans l'ordre lexicographique)}
    \end{array}\right.$  
\end{Question}
\begin{Reponse}
  \begin{Verbatim}[label=croissante(ch)]
si ch=chvide ou suite(ch)=chvide alors
  VRAI
sinon
  si premier(ch) < premier(suite(ch)) alors
    croissante(suite(ch))
  sinon
    FAUX
  finsi
finsi
  \end{Verbatim}
  \begin{description}
  \item[Terminaison et Complexité:] comme avant: $O(n)$. 
  \end{description}
\end{Reponse}

%%%%%%%% nnaturels %%%%%%%%%%%%%%%%%%%%%%%%%%%%%%%%%%%%
\begin{Question}
  $nnaturels: \left\{
    \begin{array}{l}
      \mathbb{N}\mapsto Chaine\\
      \text{retourne une chaine formée des n premiers entiers naturels}
    \end{array}\right.$  
  
  Dans un premier temps, on construira $\{n, n-1, n-2, \ldots, 3, 2, 1\}$ avant
  de construire $\{1, 2, 3, \ldots, n\}$.
\end{Question}
\begin{Reponse}
  \begin{Verbatim}[label=Version simple qui donne la liste à l'envers]
nnaturels1(n):
  si n = 0 alors chvide
           sinon adj(n, nnaturels1(n-1))
  \end{Verbatim}

  \begin{Verbatim}[label=Version trichée qui donne la chaine à l'endroit:]
nnaturels2(n):
  retourne(nnaturels1(ch))
  \end{Verbatim}
  
  Pour faire la série dans l'ordre sans tricher, il faut une fonction
  d'aide. Pour le faire trouver, on peut écrire au tableau les différents
  arguments pris par cette fonction d'aide.
  \begin{Verbatim}[label=Version avec helper:]
nnaturels3(n):
  nnaturels3_helper(1,n-1)    

nnaturels3_helper(n, todo):
  si todo = 0 alors chvide
              sinon adj(n, nnaturels3_helper(n+1,todo-1)
  \end{Verbatim}
\end{Reponse}


%%%%%%%% palindrome %%%%%%%%%%%%%%%%%%%%%%%%%%%%%%%%%%%%
\begin{Question}
  $palindrome: \left\{
    \begin{array}{l}
      Chaine\mapsto bool\acute{e}en\\
      \text{retourne VRAI si la chaine est un palindrome}
    \end{array}\right.$  

  Un palindrome se lit indifféremment de droite à gauche ou de gauche à droite.
  Exemple : « Esope reste et se repose ». On peut ignorer les espaces.
\end{Question}
\begin{Reponse}
  \begin{Verbatim}[label=palindrome(ch)]
si longueur(ch) <= 1 alors 
  VRAI
sinon
  si premier(ch) = dernier(ch) alors
    palindrome(suite(saufdernier(ch)))
  sinon 
    si premier(ch) = ' ' alors
      palindrome(suite(ch))
    sinon 
      si dernier(ch) = ' ' alors
        palindrome(saufdernier(ch))
      sinon
        FAUX
      finsi
    finsi    
  finsi
finsi    
  \end{Verbatim}
  La version simple est de ne pas ignorer les espaces, et de
  mettre un FAUX après le second «sinon» sans tester plus en avant.
\end{Reponse}

%%%%%%%% anagramme %%%%%%%%%%%%%%%%%%%%%%%%%%%%%%%%%%%%
\begin{Question}
  $anagramme: \left\{
    \begin{array}{l}
      Chaine\times Chaine\mapsto bool\acute{e}en\\
      \text{retourne VRAI si les chaines sont des anagrammes l'une de l'autre}
    \end{array}\right.$  

  Une anagramme d'un mot est un autre mot obtenu en permutant les lettres.
  Exemples: «chien» et «niche»; «baignade» et «badinage»; «Séduction»,
  «éconduits» et «on discute».
\end{Question}
\begin{Reponse}
  \begin{Verbatim}[label=anagramme(ch1\quotesinglbase ch2)]
si ch1=chvide et ch2=chvide alors
  VRAI
sinon
  si est_membre(premier(ch1), ch2) alors
    anagramme(suivant(ch1), supprime(premier(ch1), ch2))
  sinon
    FAUX
  finsi
finsi
  \end{Verbatim}
\end{Reponse}

%%%%%%%% union %%%%%%%%%%%%%%%%%%%%%%%%%%%%%%%%%%%%
\begin{Question}
  $union: \left\{
    \begin{array}{l}
      Chaine\times Chaine\mapsto Chaine\\
      \text{retourne une chaîne formée de toutes les lettres de ch1 et ch2,
        sans doublons}
    \end{array}\right.$    
  On peut supposer dans un premier temps que ch1 et ch2 ne contiennent pas de
  doublons.
\end{Question}
\begin{Reponse}
  \begin{Verbatim}[label=union(ch1\quotesinglbase ch2)]
si ch1=chvide alors
  si ch2=chvide alors
    chvide
  sinon
    union(ch2,chvide) -- Pour virer les doublons de ch2
  finsi
sinon
  si est_membre(premier(ch1), suite(ch1)) ou est_membre(premier(ch1), ch2) alors
    union(suite(ch1),ch2)
  sinon
    adj(premier(ch1),union(suite(ch1),ch2))
  finsi
finsi    
  \end{Verbatim}
\end{Reponse}

%%%%%%%% difference %%%%%%%%%%%%%%%%%%%%%%%%%%%%%%%%%%%%
\begin{Question}
  $difference: \left\{
    \begin{array}{l}
      Chaine\times Chaine\mapsto Chaine\\
      \text{retourne toutes les lettres de ch1 ne faisant
        pas partie de ch2}
    \end{array}\right.$    
\end{Question}
\begin{Reponse}
  \begin{Verbatim}[label=difference(ch1\quotesinglbase ch2)]
si ch2 = chvide alors
  ch1
sinon 
  si ch1 = chvide alors
    chvide
  sinon
    si est_membre(premier(ch1),ch2) alors
      difference(suite(ch1),ch2)
    sinon
      adj(premier(ch1), difference(suite(ch1),ch2))
    finsi
  finsi
finsi    
  \end{Verbatim}
\end{Reponse}

\end{document}
%%% Local Variables:
%%% coding: utf-8
