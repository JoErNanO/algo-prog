\documentclass[10pt]{article}\usepackage[correction,nu]{esial}%[correction]{esial}

\usepackage[utf8]{inputenc}
\usepackage{url}
\usepackage{amstext,amsmath,amsfonts}
\usepackage{fancyvrb}

\TOP\unA
\title{TP5: Récursivité (récipients)}

\begin{document}
\maketitle


L'objectif de ce nouveau TDP est de résoudre un nouveau problème par
backtracking. La spécificité de ce problème-ci est que si vous agissez sans
précaution, votre programme peut entrer en boucle infinie...

\Exercice\textbf{Présentation du problème des récipients.}

On dispose d'un certain nombre de récipients dont on connaît la capacité, et
d'une fontaine d'eau. On cherche quels sont les transvasements à réaliser pour
passer faire en sorte que l'un des récipients contienne une quantité d'eau
donnée. Seules les opérations suivantes sont autorisées:
\begin{enumerate}
\item[A.] Remplir complètement un récipient depuis la fontaine;
\item[B.] Vider complètement un récipient dans la fontaine;
\item[C.] Transvaser un récipient dans un autre jusqu'à ce que la source soit
  complètement vide ou que la destination soit complètement pleine.
\end{enumerate}

\noindent$\triangleright$ \textbf{Exemple.} On suppose avoir deux récipients de
capacité respective 5 et 3. On veut mesurer un volume de 4. D'après une
situation initiale où les deux récipients sont vides, notée $(0,0)$, les
opérations suivantes permettent d'y parvenir.

\noindent\begin{minipage}{.35\linewidth}
  \begin{itemize}
  \item Remplir A à la fontaine: (5,0)
  \item Transvaser A dans B: (2,3)
  \item Vider le contenu de B: (2,0)
  \item Transvaser A dans B: (0,2)
  \item Remplir A à la fontaine: (5,2)
  \item Transvaser A dans B: (4,3)
  \item On a bien 4 unités dans A.
  \end{itemize}  
\end{minipage}~\begin{minipage}{.6\linewidth}
  \centerline{\includegraphics[width=\linewidth]{fig/recipients-example.fig}}
\end{minipage}

\Question Pouvez-vous trouver une instance de ce problème n'admettant pas de
solution?
\begin{Reponse}
  \noindent
  Ben ca va être plus dur, si, au choix:
  \begin{itemize}
  \item Si la cible est plus grande que n'importe quel récipient
  \item Si toutes les capacités sont égales les unes aux autres, et différentes
    de la cible cherchée
  \end{itemize}
\end{Reponse}

\Exercice\textbf{Réflexions sur le codage.}  Pour simplifier, nous vous
proposons d'organiser votre code de la façon suivante. Il s'agit d'un schéma
très classique et relativement proche de ce que nous avons mis en œuvre
précédement pour le sac à dos ou les pyramides.

\noindent\includegraphics[width=\linewidth]{diagramme-classe.fig}

\Question Donnez le (pseudo-)code de la méthode \texttt{Recipient.transvase(Recipient)}.

\begin{Reponse}
  \begin{Verbatim}[gobble=4]
    transvase(src,dst)
      quantite = Min(src.contenu,dst.capacite-dst.contenu)
      src.contenu -= quantite
      dst.contenu += quantite
  \end{Verbatim}
\end{Reponse}
\Question Implémentez ces classes, à l'exception de la méthode solve() du
solveur, qui constitue le cœur du TP et sera implémenté plus tard. Dans la
classe de test, utilisez l'instance du problème présentée dans l'exemple
ci-dessus. L'intéret est que vous avez la garantie qu'une solution existe pour
cette instance.

\Exercice\textbf{Réflexions algorithmiques.}  En absence de meilleure idée, nous
allons établir une recherche exaustive des solutions de transvasement jusqu'à
trouver une situation où l'un des récipients contienne la bonne quantité de
liquide. Comme vous vous en doutez, nous allons le faire par backtracking.

\Question Quel est le paramètre de récursion? Quels sont les cas triviaux?

\begin{Reponse}
  Pour le paramètre, on a pas de bonne idée selon les données. La seule
  possibilité est le nombre de transvasements faits jusqu'à présent. 
  Pour les cas triviaux, c'est si un récipient contient ce qu'on veut.
\end{Reponse}

\medskip Comme souvent lors d'une recherche combinatoire, nous allons donc
utiliser une récursion dont chaque étage est une boucle parcourant toutes les
possibilités existantes. Le pseudo-code serait quelque chose comme ça:
\begin{Verbatim}[gobble=2]
  fonction_recursive(parametres)
    Si je suis sur un cas trivial, j'arrête
    Pour chaque décision D que je peux prendre maintenant
      appliquer D
      fonction_recursive(paramètres modifiés)
      annuler les changements dûs à D
\end{Verbatim}

\Question Relisez le cours à propos du placement des reines, le code que vous
avez écrit pour le sac à dos et celui pour les pyramides pour voir comment ce
pseudo-code était mis en pratique dans chaque cas.

\Question Écrire la liste des actions autorisées à chaque étape de la recherche
exaustive.\\ \textit{Indication:} il y a sans doute deux boucles \texttt{for}
imbriquées, et un \texttt{if} dedans.

\begin{Reponse}
  \begin{Verbatim}[gobble=4]
    for (a=0;a<=NB_RECIPIENTS;a++) { // on choisit un recipient source 
      for (b=0;b<=NB_RECIPIENTS;b++) { // et un recipient cible ... 
        // Si a==NB_RECIPIENTS, on remplit depuis la fontaine
        // Si b==NB_RECIPIENTS, on vide dans la fontaine
        if (a != b) {
          // Le corps ici
        }
      }
    }
  \end{Verbatim}
\end{Reponse}

\Question Écrivez le pseudo-code de la fonction récursive en combinant les deux
questions précédentes. Ne cherchez pas encore à résoudre le problème posé par
la ligne ``annuler les changements dûs à D'' dans le pseudo-code.

\begin{Reponse}
  \begin{Verbatim}[gobble=4]
    boolean solve(int n) {
      for (src=0;src<=NB_RECIPIENTS;src++) { // on choisit un recipient source 
        for (dst=0;dst<=NB_RECIPIENTS;dst++) { // et un recipient cible ... 
          if (src != dst) {
            // Choisi et applique une décision
            if (src==NB_RECIPIENTS) { // on remplit depuis la fontaine
              recipients[dst].remplir();
            } else if (dst==NB_RECIPIENTS) { // on vide dans la fontaine
              recipients[dst].vider();
            } else {
              recipients[src].transvase(recipients[dst]);
            }
            
            solve(n+1);

            // TODO: annule la décision
          }
        }
      }      
    }
  \end{Verbatim}
\end{Reponse}

\medskip
Il s'agit maintenant de sauvegarder l'état avant les modifications,
et de le restaurer après la récursion. Le plus simple pour cela est d'ajouter
un \textit{copy-constructor} à la classe Recipient, c'est à dire un
constructeur prenant un autre récipient en argument et faisant en sorte que
l'objet nouvellement créé soit une copie conforme de l'argument. 

\Question Écrivez ce constructeur (sans ajouter d'autre méthode ni rendre les
champs publiques), et utilisez le pour finir la fonction solve.

\begin{Reponse}
  Souvent, ca leur pose problème car ils pense qu'un champ privé est privé à
  une instance d'objet. Et non, la frontière est classe par classe et je peux
  tout à fait aller taper sur les champs privés d'un autre objet, du moment
  qu'ils sont de la même classe que moi.

  \begin{Verbatim}[gobble=4]
    public Recipient(Recipient other) {
      this.capacite = other.capacite;
      this.contenu = other.contenu;
    }
  \end{Verbatim}
\end{Reponse}

\medskip
Codé comme cela, notre code ``fonctionnerait'', mais entrerait en
boucle infinie. À chaque étage de la récursion, nous déciderions de transvaser
le premier récipient dans le second (ce qui ne sert à rien vu que les deux sont
vides), avant de plonger un étage plus bas dans la récursion. À l'infini.
\Question Pour résoudre ce problème, modifiez votre pseudo-code pour arrêter la
recherche après un nombre donné de transvasements.

\begin{Reponse}
  On fait tjs une recherche en profondeur, mais bornée cette fois, donc ca va
  aller: le code va faire n'imp sur N étapes, puis tester des choses un peu
  plus variées en remontant.

  Il faut ajouter la profondeur max pour \texttt{new Solveur()} dans la classe
  de tests, ou bien passer à \texttt{solve()} le nombre d'étapes qu'il a encore
  le droit de faire. Et ca nous fait un nouveau cas trivial: si la profondeur
  courante dépasse la profondeur autorisée (ou si le nb d'étapes restantes
  arrive à 0), on coupe.
\end{Reponse}

\Exercice\textbf{Première implémentation (bounded depth first).}  
\Question Vous ne devriez pas être loin d'une solution fonctionnelle, et
implémenter votre pseudo-code ne devrait pas poser beaucoup de problème.

La seule chose manquante est une idée pour afficher le transvasement ayant mené
à la solution lorsque vous en trouvez une. Pour cela, on peut construire lors
de la descente une chaînes de caractères décrivant les opérations
effectuées. Une autre idée est d'ajouter à la fonction récursive un paramètre
une liste d'opérations de transvasement et l'afficher quand on trouve, mais
cela demande de définir une classe représentant ces opérations.

\Question Retrouvez la solution à l'instance du problème donnée en exemple
ci-dessus. 

\Question Trouvez une solution si les capacités valent \{8,5,3\} et
que l'on cherche à obtenir 6 volumes en moins de 3 transvasements. Si
c'est impossible, augmentez progressivement le nombre de
transvasements autorisés au maximum jusqu'à trouver une solution.
Répondez aux questions suivantes le temps que votre programme trouve
une réponse.

\begin{Reponse}
  Je l'ai pas codé, j'ai pas la solution. J'espère que cette instance est pas
  triviale sinon, c'est dommage. Voici ce que j'ai trouvé sur internet
  (attention, ca semble juste, mais j'ai lu rapidement)

  \VerbatimInput{idee_solution}
\end{Reponse}

\Question Même question si les capacités valent \{100, 25,24\} et que
l'on cherche à obtenir 42 volumes\footnote{D'après une instance
trouvée par Oswald Hounkonnou, promo 2012.}. Il est probable que votre
programme ne parvienne pas à la solution tant que vous n'aurez pas
implémenté les améliorations de l'exercice suivant.


\smallskip\noindent\setcounter{Question}{0}\hspace{-1.3em}$\bigstar$
\textbf{Plus de contraintes pour plus de backtracking.} Notre approche
``recherche en profondeur, avec profondeur maximale'' est intéressante en ceci
qu'elle permet d'éviter la boucle infinie consistant à réaliser la première
action autorisée à chaque étage de récursion, même si cela ne mene nul part
(cf. question 6 ci-dessus).

En revanche, son gros défaut est qu'il est difficile de déterminer la valeur à
utiliser comme profondeur maximale. S'il est trop petit, on ne trouvera pas la
solution. S'il est trop grand, l'espace grandit très (trop) vite. De plus, si
on relance le programme avec une valeur plus grande, tous les calculs effectués
avec une profondeur inférieure sont refaits. Cette situation est clairement
sous-optimale\ldots

On peut constater que notre programme effectue certes une quantité infinie
d'opérations, mais qu'il n'existe qu'un nombre fini de situations. En effet,
pour $n$ récipients de capacité $c_i$ chacun, nous savons que le premier
récipient contient 0 unité ou bien 1 unité ou bien 2 \ldots ou bien $c_1$. De
même, le nombre de façon de remplir chacun des autres récipients est borné. On
en déduit que le nombre de plateaux (de remplissage de l'ensemble des
récipients) n'est pas infini. Notre code fait donc des choses inutiles.

Pour changer cela, nous allons faire en sorte de ne parcourir que des solutions
originales (jamais rencontrées auparavent), et couper par backtracking si on
rencontre à nouveau une solution déjà vue. Pour déterminer si la situation
actuelle a déjà été rencontrée auparavent, constatons tout d'abord qu'une
situation est un vecteur de nombres, indiquant le remplissage de chaque
récipients.

\Exercice\textbf{Stockage par liste de vecteurs.} Une première approche
consiste à faire une liste de tous les vecteurs de valeur rencontrés. À chaque
fois que l'on rencontre une nouvelle situation (au début de notre fonction
récursive), on parcourt la liste pour voir si le vecteur courant est original,
et on coupe court à la recherche si non.

\Question Vous pouvez implémenter cette solution, ou constater sa mauvaise
efficacité (tant en temps de calcul qu'en consomation mémoire).

\Exercice \textbf{Stockage par hachage.} Avec un peu plus de connaissances en
Java que ce qui est demandé en TOP, une autre approche est d'utiliser une
structure de données classique pour profiter des bonnes propriétés des tables
de hachage (qui sont au programme du module de SD). Il suffit créer une
variable de type \texttt{HashTree<String,Boolean>}, et d'utiliser ensuite les
fonctions permettant de d'insérer un élément, et chercher si un élément donné
existe dans la table. La clé des éléments sera le résultat de la méthode
\texttt{toString()} appliquée à la solution courante. 

\Question Implémentez cette solution en vous appuyant sur la documentation des
\texttt{HashTree}.

\Question Discutez l'efficacité de cette solution (en particulier en terme de
mémoire).
\begin{Reponse}
  Ben c'est pas bon : on crée des chaines à tout bout de champs et on les
  stoque. On va faire sauter la mémoire. 

  En plus, on hache les chaines de représentation. On risque d'avoir pleins de
  conflits vu que les chaines se ressemblent toutes. Mais ca, vu qu'ils ont pas
  vu ce qu'est une table de hachage, on peut pas leur demander.
\end{Reponse}

\Exercice\textbf{Stockage par tableau booléen.} Avec un minimum de connaissances
mathématiques, une autre approche est de coder chaque vecteur de remplissage
sous forme d'un entier unique. Il faut pour cela trouver une fonction allant de
l'ensemble des vecteurs possibles dans l'ensemble des entiers. On utilisera la
fonction \texttt{int hashCode()} pour cela.

\Question Quelle propriété doit avoir cette fonction?
\begin{Reponse}
  Injective (il me semble, je me goure tjs). Il faut que $\forall x,y, (x\neq y)
  \Rightarrow (code(x)\neq code(y))$
\end{Reponse}

Pour construire une telle fonction, on peut par exemple multiplier chaque
élément du vecteur par un nombre premier différent.  Par exemple, si le vecteur
est de longueur 3, on peut utiliser les nombres premiers suivants:
\{3,5,7\}. Ainsi, le vecteur [1,14,4] sera représenté par l'entier $3\times
1+14\times 5+7\times 4=101$. Il suffit alors de disposer d'un tableau booléen
nommé par exemple \texttt{dejaVu}, et de stocker sous \texttt{dejaVu[101]} si
l'on a déjà vu le vecteur [1,14,4].

\Question Implémenter cette solution.

\Question Discutez l'efficacité de cette solution. Comment peut-on améliorer
les choses?
\begin{Reponse}
  Le hachage va mieux se passer, et on a beaucoup moins de chaines
  construites. C'est bien mieux. Mais on se trimbale maintenant avec un tableau
  de booléens de dimension de folie.

  On peut améliorer en stockant l'information sous forme d'une liste (chainée)
  triée des valeurs déjà rencontrées. Le stockage sera plus efficace, la
  recherche est en $O(\log(n))$, et l'insertion en $0(1)$.

  Le mieux est de faire un stockage creux du vecteur (c'est comme ca qu'on dit
  en calcul scientifique, rien de grave). On économise des pointeurs dans la
  liste ci-dessus en stockant des vecteurs dans chaque case du tableau, et en
  fusionnant les cellules adjacentes dans le meme vecteur.

  $(1) \leftrightarrow (2) \leftrightarrow (3) \leftrightarrow (76)
  \leftrightarrow (77) \leftrightarrow 205$ devient $(1,2,3) \leftrightarrow
  (76,77) \leftrightarrow 205$, ce qui sauve une poignée de vecteurs. Ca risque
  d'être précieux si la structure grandit.
\end{Reponse}

\end{document}
%%% Local Variables:
%%% coding: utf-8
