\documentclass[10pt]{article}\usepackage[correction,nu]{esial}%[correction]{esial}
%\documentclass[10pt]{article}\usepackage[nu]{esial}%[correction]{esial}

\usepackage[utf8]{inputenc}
\usepackage{url}
\usepackage{amstext,amsmath,amsfonts}
\usepackage{fancyvrb,wrapfig}
\hypersetup{hidelinks=true}

\TOP\unA
\title{TP5: Récursivité (récipients)}

\begin{document}
\maketitle

L'objectif de ce nouveau TDP est de résoudre un nouveau problème par
backtracking. La spécificité de ce problème-ci est que si vous agissez sans
précaution, votre programme peut entrer en boucle infinie...

\Exercice\textbf{Présentation du problème des récipients.}

On dispose d'un certain nombre de récipients dont on connaît la capacité, et
d'une fontaine d'eau. On cherche quels sont les transvasements à réaliser pour
passer faire en sorte que l'un des récipients contienne une quantité d'eau
donnée. Seules les opérations suivantes sont autorisées:
\begin{enumerate}
\item[A.] Remplir complètement un récipient depuis la fontaine;
\item[B.] Vider complètement un récipient dans la fontaine;
\item[C.] Transvaser un récipient dans un autre jusqu'à ce que la source soit
  complètement vide ou que la destination soit complètement pleine.
\end{enumerate}

\noindent$\triangleright$ \textbf{Exemple.} On suppose avoir deux récipients de
capacité respective 5 et 3. On veut mesurer un volume de 4. D'après une
situation initiale où les deux récipients sont vides, notée $(0,0)$, les
opérations suivantes permettent d'y parvenir.

\noindent\begin{minipage}{.35\linewidth}
  \begin{itemize}
  \item Remplir A à la fontaine: (5,0)
  \item Transvaser A dans B: (2,3)
  \item Vider le contenu de B: (2,0)
  \item Transvaser A dans B: (0,2)
  \item Remplir A à la fontaine: (5,2)
  \item Transvaser A dans B: (4,3)
  \item On a bien 4 unités dans A.
  \end{itemize}  
\end{minipage}~\begin{minipage}{.6\linewidth}
  \centerline{\includegraphics[width=\linewidth]{fig/recipients-example.fig}}
\end{minipage}

\Question Pouvez-vous trouver une instance de ce problème n'admettant pas de
solution?
\begin{Reponse}
  \noindent
  Cela va être plus dur, si, au choix:
  \begin{itemize}
  \item Si la cible est plus grande que n'importe quel récipient
  \item Si toutes les capacités sont égales les unes aux autres, et différentes
    de la cible cherchée
  \end{itemize}
  Attention, il n'y a pas d'histoires de PGCD ici, car ce n'est pas
  multiplicatif, mais additif.
\end{Reponse}

\Exercice\textbf{Réflexions sur le codage.}\\
La première étape est de réfléchir à comment représenter ce problème dans notre
programme, puis à écrire les fonctions d'aide permettant par exemple de remplir
un objet à la fontaine, de transvaser un objet dans un autre, et de vider un
objet dans la fontaine. Le plus simple est probablement de s'inspirer de ce que
nous avons fait dans les TP précédents.

\begin{Reponse}
  Dans ma solution, j'ai pris 2 tableaux et un entier pour cela. Il faut donc
  les amener à découvrir ceci:
  \begin{itemize}
  \item un tableau \textbf{capa} dénote la capacité maximale de chaque récipient;
  \item un tableau \textbf{ctn} dénote le contenu actuel de chaque récipient;
  \item un entier \textbf{cible} dénote la cible à atteindre.
  \end{itemize}

  \newcommand*\FancyVerbStartString{// BEGIN HELPERS}
  \newcommand*\FancyVerbStopString{// END HELPERS}
  \VerbatimInput{src-scala/boundedDFS.scala}

\end{Reponse}

% Pour simplifier, nous vous proposons d'organiser votre code de la façon
% suivante. Il s'agit d'un schéma très classique et relativement proche de ce que
% nous avons mis en œuvre précédement pour le sac à dos ou les pyramides.
% %
% \noindent\includegraphics[width=\linewidth]{fig/diagramme-classe.fig}

% \Question Donnez le (pseudo-)code de la méthode \texttt{Recipient.transvase(Recipient)}.

% \begin{Reponse}
%   \begin{Verbatim}[gobble=4]
%     transvase(src,dst)
%       quantite = Min(src.contenu,dst.capacite-dst.contenu)
%       src.contenu -= quantite
%       dst.contenu += quantite
%   \end{Verbatim}
% \end{Reponse}
% \Question Implémentez ces classes, à l'exception de la méthode solve() du
% solveur, qui constitue le cœur du TP et sera implémenté plus tard. Dans la
% classe de test, utilisez l'instance du problème présentée dans l'exemple
% ci-dessus. L'intéret est que vous avez la garantie qu'une solution existe pour
% cette instance.

\Exercice\textbf{Parcours exhaustif simple (mais borné).} \\
En absence de meilleure idée, nous allons établir une recherche exaustive des
solutions de transvasement jusqu'à trouver une situation où l'un des récipients
contienne la bonne quantité de liquide. Comme vous vous en doutez, nous allons
le faire par backtracking.

\Question Quel est le paramètre de récursion? Quels sont les cas triviaux?

\begin{Reponse}
  Pour le paramètre, on a pas de bonne idée selon les données. La seule
  possibilité est le nombre de transvasements faits jusqu'à présent. 
  Pour les cas triviaux, c'est si un récipient contient ce qu'on veut. C'est pas
  énorme, mais c'est tout ce qu'on a pour l'instant.
\end{Reponse}

\medskip Comme souvent lors d'une recherche combinatoire, nous allons donc
utiliser une récursion dont chaque étage est une boucle parcourant toutes les
possibilités existantes. Le code ressemble à cela:
\begin{Verbatim}[gobble=2]
  fonction_recursive(parametres)
    Si je suis sur un cas trivial, j'arrête
    Pour chaque décision D que je peux prendre maintenant
      appliquer D
      fonction_recursive(paramètres modifiés)
      annuler les changements dûs à D
\end{Verbatim}

\Question Relisez le cours à propos du placement des reines, le code que vous
avez écrit pour le sac à dos et celui pour les pyramides pour voir comment cette
idée était mise en pratique dans chaque cas.

\Question Écrire la liste des actions autorisées à chaque étape de la recherche
exhaustive.\\ \textit{Indication:} il y a sans doute deux boucles \texttt{for}
imbriquées, et un \texttt{if} dedans.

\begin{Reponse}
  Le truc est de donner un numéro à la fontaine, en fait. 
  \begin{Verbatim}[gobble=4]
    for (src <- 0 to NB_RECIPIENTS) {   // on choisit un récipient source 
      for (dst <- 0 to NB_RECIPIENTS) { // et un récipient cible ... 
        // Si src==NB_RECIPIENTS, on remplit depuis la fontaine
        // Si dst==NB_RECIPIENTS, on vide dans la fontaine
        if (src != dst) {
          // Le corps ici
        }
      }
    }
  \end{Verbatim}
  % \begin{Verbatim}[gobble=4]
  %   for (a=0;a<=NB_RECIPIENTS;a++) { // on choisit un récipient source 
  %     for (b=0;b<=NB_RECIPIENTS;b++) { // et un récipient cible ... 
  %       // Si a==NB_RECIPIENTS, on remplit depuis la fontaine
  %       // Si b==NB_RECIPIENTS, on vide dans la fontaine
  %       if (a != b) {
  %         // Le corps ici
  %       }
  %     }
  %   }
  % \end{Verbatim}
\end{Reponse}

\Question Écrivez le code de la fonction récursive en combinant les deux
questions précédentes. Ne cherchez pas à résoudre le problème posé par la ligne
``annuler les changements dûs à D'' pour l'instant.

\begin{Reponse}
  Le code de toutes les réponses est une fois pour toute, un peu plus bas.
  % \begin{Verbatim}[gobble=4]
  %   boolean solve(int n) {
  %     for (src=0;src<=NB_RECIPIENTS;src++) { // on choisit un recipient source 
  %       for (dst=0;dst<=NB_RECIPIENTS;dst++) { // et un recipient cible ... 
  %         if (src != dst) {
  %           // Choisi et applique une décision
  %           if (src==NB_RECIPIENTS) { // on remplit depuis la fontaine
  %             recipients[dst].remplir();
  %           } else if (dst==NB_RECIPIENTS) { // on vide dans la fontaine
  %             recipients[dst].vider();
  %           } else {
  %             recipients[src].transvase(recipients[dst]);
  %           }
            
  %           solve(n+1);

  %           // TODO: annule la décision
  %         }
  %       }
  %     }      
  %   }
  % \end{Verbatim}
\end{Reponse}

\Question %
Il s'agit maintenant de sauvegarder l'état avant les modifications, et de le
restaurer après la récursion.  Le plus simple est de stocker dans des variables
spécifiques le contenu des récipients modifiés à cet étage de récursion, puis de
restaurer ces valeurs après la récursion. Modifiez votre code en ce sens.

\begin{Reponse}
  Oui, ce viol de l'encapsulation a largement de quoi faire faire un arrêt à un
  grand maître du design objet. Mais c'est du fonctionnel, là. Et je veux juste
  les faire trouver l'algo. Donc, pas de copy-constructor et tout le
  tintouin. Mais faut admettre que c'est bien plus simple ainsi (cf. plus bas). 

  Si vous voulez vraiment savoir, les objets me manquent pour représenter la
  solution dans sa globalité. Mais puisqu'on n'en a pas, autant en profiter.
\end{Reponse}

% Le plus simple pour cela est d'ajouter un \textit{copy-constructor} à la classe
% Recipient, c'est à dire un constructeur prenant un autre récipient en argument
% et faisant en sorte que l'objet nouvellement créé soit une copie conforme de
% l'argument.

% \Question Écrivez ce constructeur (sans ajouter d'autre méthode ni rendre les
% champs publiques), et utilisez le pour finir la fonction solve.

% \begin{Reponse}
%   Souvent, ca leur pose problème car ils pense qu'un champ privé est privé à
%   une instance d'objet. Et non, la frontière est classe par classe et je peux
%   tout à fait aller taper sur les champs privés d'un autre objet, du moment
%   qu'ils sont de la même classe que moi.

%   \begin{Verbatim}[gobble=4]
%     public Recipient(Recipient other) {
%       this.capacite = other.capacite;
%       this.contenu = other.contenu;
%     }
%   \end{Verbatim}
% \end{Reponse}

\Question %
En l'état, ce code ``fonctionnerait'', mais entrerait en boucle infinie. À
chaque étage de la récursion, nous déciderions de transvaser le premier
récipient dans le second (ce qui ne change rien puisque que les deux sont
vides), avant de plonger un étage plus bas dans la récursion. À l'infini.  Pour
résoudre cela, modifiez votre code pour ne pas effectuer les opérations qui ne
mènent visiblement à rien. Attention à ne pas couper l'exploration trop
brutalement, car sinon on risque de rater des branches importantes.
\begin{Reponse}
  Les coups inutiles à éviter sont: 
  \begin{itemize}
  \item Remplir à la fontaine un récipient déjà plein
  \item Vider dans la fontaine un objet déjà vide
  \item Transvaser depuis un objet vide
  \item Transvaser dans un objet plein
  \end{itemize}
  Quand l'une de ces situations se produit, il faut ne pas faire la modification
  de l'état, ni l'appel récursif, ni la restauration de l'état initial. Mais
  attention, il ne faut pas faire return car sinon on coupe les deux boucles
  for, ce qui est bien trop brutal.
\end{Reponse}

\Question Malheureusement, les boucles infinies sont toujours possible malgré
cette modification. Considérez l'historique suivant: remplir(0);
transvaser(0,1); transvaser(1,0); transvaser(0,1); transvaser(1,0);
transvaser(0,1); transvaser(1,0); transvaser(0,1); transvaser(1,0);
transvaser(0,1); \ldots

Pour se prémunir contre ce problème, nous allons borner la profondeur de
recherche. C'est-à-dire que vous devez arrêter la recherche après une quantité
prédéterminée de transvasements.

\begin{Reponse}
  On fait toujours une recherche en profondeur, mais bornée cette fois, donc ça
  va aller: le code va faire n'importe quoi sur N étapes, remonter, puis tester
  des choses un peu plus variées sur les autres branches.

  Il faut juste préciser la profondeur max lors de l'appel initial.
  % pour \texttt{new Solveur()} dans la classe
  % de tests, ou bien passer à \texttt{solve()} le nombre d'étapes qu'il a encore
  % le droit de faire.
  Et ça nous fait un nouveau cas trivial: si le nombre d'étapes encore à
  réaliser arrive à 0, on coupe.
\end{Reponse}

\Question Ajoutez maintenant ce qu'il faut pour afficher la séquence
d'opérations menant à la solution une fois que vous l'avez trouvé. Comme
souvent, plusieurs solutions sont possibles: on peut soit afficher les coups
effectués (en sens inverse) lors de la remontée récursive, soit construire une
chaîne de caractères (ou autres) lors de la descente décrivant les coups
réalisés avec un accumulateur.

\begin{Reponse}
  J'ai fait le crado, j'ai fait en sorte que la fonction récursive renvoie un
  boolean indiquant si elle a trouvé une solution convenable. Et si oui, chaque
  étage affiche sa dernière modif lors de la remontée récursive. Il faut ensuite
  lire de bas en haut.

  \newcommand*\FancyVerbStartString{// BEGIN SOLVE}
  \newcommand*\FancyVerbStopString{// END SOLVE}
  \VerbatimInput{src-scala/boundedDFS.scala}
\end{Reponse}

\Question Sur machine, retrouvez la solution donnée en exemple d'introduction de
ce sujet, où les capacités sont \{3 et 5\} pour une cible de 4.

\Question Quel est le nombre minimal de transvasements pour retrouver 6 avec des
récipients de tailles \{8, 5, 3\}?

\begin{Reponse}
  Si on demande à profondeur 3, il n'y a pas de solution, mais à profondeur 4,
  on trouve ceci (une fois remis dans l'ordre):
  \begin{Verbatim}[gobble=4]
    Remplir 0 a la fontaine. Situation après ce coup: {8, 0, 0}    
    Transvaser 0 dans 1.     Situation après ce coup: {3, 5, 0}
    Transvaser 1 dans 2.     Situation après ce coup: {3, 2, 3}
    Transvaser 2 dans 0.     Situation après ce coup: {6, 2, 0}
  \end{Verbatim}
\end{Reponse}

\Question Même question pour retrouver 42 avec des récipients de tailles \{100,
24, 25\}\footnote{Récipients=\{100, 24, 25\}; Cible=42: Instance proposée par
  Oswald Hounkonnou, promo 2012.}. Cette instance du problème est agaçante, car
on peut la résoudre de tête puisque la séquence ``vider 24; remplir 25;
transvaser 25 dans 24'' permet d'obtenir une unité. Il suffit alors de la
répéter 42 fois pour trouver la cible. Mais notre programme est trop long pour
parvenir à trouver une solution de profondeur 196 comme celle-ci.

\begin{Exercice}
  \textbf{Plus de contraintes pour couper plus de branches.} 
  
  Nous avons dû borner la profondeur maximale de la recherche car notre
  programme effectue parfois une quantité infinie d'opérations. Cela se produit
  quand il trouve un cycle d'actions consistant à faire et défaire, comme
  ``A$\rightarrow$B ; B$\rightarrow$A''. Pourtant, il n'existe qu'un nombre fini
  de situations. En effet, pour $n$ récipients de capacité $c_i$ chacun, nous
  savons que le premier récipient contient 0 unité ou bien 1 unité ou bien 2
  \ldots ou bien $c_1$. De même, le nombre de façon de remplir chacun des autres
  récipients est borné. On en déduit que le nombre de plateaux (de remplissage
  de l'ensemble des récipients) n'est pas infini. Notre code fait donc des
  choses inutiles.

  Pour changer cela, nous allons faire en sorte de ne parcourir que des
  solutions originales (jamais rencontrées auparavant), et couper l'exploration
  si l'on rencontre à nouveau une solution déjà vue. Pour déterminer si la
  situation actuelle a déjà été rencontrée auparavant, il suffit de stocker dans
  une liste toutes les situations rencontrées jusque là.

  \Question Écrivez les deux fonctions suivantes, qui teste l'égalité entre deux
  tableaux de même taille et qui teste si un élément donné est dans la liste
  donnée. \texttt{contient()} utilise naturellement \texttt{egal()}.
\begin{Verbatim}
  def egal(A:Array[Int], B:Array[Int]):Boolean = {...}
  def contient(liste:List[Array[Int]], elm:Array[Int]):Boolean= {...}
\end{Verbatim}

La complexité asymptotique de ces fonctions est clairement un grand polynôme,
mais en pratique, elles s'avèrent suffisantes dans certains cas.

\Question Modifiez votre code pour stocker la liste de tous les états visités
dans une variable globale. Il ne faut faire l'appel récursif que si l'état
actuel est nouveau. Sinon, il faut couper l'exploration. Il n'est plus
nécessaire de borner la profondeur de parcours, puisque les branches menant à
des cas déjà visités seront coupées, rendant impossible toute recherche infinie.

\begin{Reponse}
  \newcommand*\FancyVerbStartString{// BEGIN SOLVE}
  \newcommand*\FancyVerbStopString{// END SOLVE}
  \VerbatimInput{src-scala/memoizedDFS.scala}
\end{Reponse}

\Question Sur machine, testez votre code sur l'instance d'Hounkonnou. Une
solution est trouvée en moins d'une minute, bien que la fonction
\texttt{contient} soit en $O(n^3)$. Ce bon résultat est probablement dû à la
chance, où nous n'avons pas besoin d'explorer toutes les branches pour trouver
la solution.

En revanche, la solution trouvée de cette façon implique 259 opérations alors
que l'on sait qu'il existe une solution en 196 opérations.
\end{Exercice}


  % C'est le fichier memoizedDFS.scala de l'archive. Ça marche plutôt bien en
  % pratique car l'instance d'Émmanuel est conçue pour exploser en largeur, mais
  % elle reste de profondeur assez raisonnable. L'idée de l'instance est que ces 4
  % nombres font partie de la suite de Fibbonacci. On sait donc que le tout
  % dernier n'est pas impliqué dans le chemin le plus court. Il est juste là pour
  % augmenter la largeur. L'avant dernier n'est là quant à lui que pour stocker le
  % reste des opérations sur les deux premiers. Résultat, là où y'a 510 états
  % distincts accessibles à la profondeur 10 de l'instance d'Hounkonnou, il y en a
  % 2157 pour celle-ci. 

  % J'imagine cependant que ça marche nettement moins bien si on place les valeurs
  % dans un autre ordre.


\ifcorrection{}{\newpage}
\begin{Exercice}\textbf{Parcours en largeur.}

\noindent
\begin{minipage}{.74\linewidth}
  On peut constater que même dans les cas où il fonctionne bien, notre code fait
  souvent du travail inutile. Par exemple, si on lui demande de chercher
  l'exemple de l'énoncé, il trouve la solution ci-contre en 10 coups alors qu'on
  en connaît une en 7 coups seulement. C'est que les cinq premiers coups sont
  une façon bien compliquée de remplir le récipient B!

  Le problème, qui explique également que la solution trouvée pour l'instance
  d'Hounkonnou ne soit pas optimale, vient de l'ordre de parcours des solutions
  possibles. Nous avons choisi un ordre qui s'appelle classiquement ``en
  profondeur'' (depth first en anglais). Cela explique que notre programme
  trouve d'abord la solution ci-contre avant de trouver la solution de l'énoncé,
  qui est certes plus courte mais rencontrée plus tard lors d'un parcours en
  profondeur.
\end{minipage}\hfill\begin{minipage}{.24\linewidth}
  \begin{tabular}{|c|c|}\hline
    Coup&Résultat\\\hline
    Remplir A         & (5, 0)\\\hline
    A $\rightarrow$ B & (0, 5)\\\hline
    Remplir A         & (5, 5)\\\hline
    A $\rightarrow$ B & (3, 7)\\\hline
    Vider A           & (0, 7)\\\hline
    B $\rightarrow$ A & (5, 2)\\\hline
    Vider A           & (0, 2)\\\hline
    B $\rightarrow$ A & (2, 0)\\\hline
    Remplir B         & (2, 7)\\\hline
    B $\rightarrow$ A & (5, 4)\\\hline
  \end{tabular}
\end{minipage}

Une piste pour éviter ce problème, proposée par Arthur Brongniart (promo 2013)
est de réaliser un parcours en largeur au lieu d'un parcours en profondeur.  Il
s'agit d'explorer entièrement un étage de l'arbre avant de descendre au niveau
suivant. Ce type de parcours est un peu plus compliqué à réaliser.

Pour cela, chaque étage de la récursion prend la liste de toutes les situations
que l'on peut atteindre à la profondeur précédente, et construit la liste de
celles que l'on peut atteindre avec une étape de plus.

\Question Écrivez une telle fonction récursive qui calcule la profondeur
minimale permettant de résoudre une instance donnée du problème. Il semble
difficile d'afficher les coups ayant mené à la solution dans le cas d'un
parcours en largeur pour l'instant. Contentez vous d'afficher la profondeur
minimale nécessaire. De même, n'apportez aucune optimisation pour l'instant.

\begin{Reponse}
  \newcommand*\FancyVerbStartString{// BEGIN CHERCHE}
  \newcommand*\FancyVerbStopString{// END CHERCHE}
  \VerbatimInput{src-scala/largeur.scala}
\end{Reponse}

\Question %
On voit clairement que cette solution correcte va trouver la réponse la plus
courte, puisqu'elle évalue les solutions dans l'ordre de leur longueur. En
revanche, elle ne permet pas de trouver la solution pour l'instance Hounkonnou
car l'ordinateur arrive à court de mémoire pour stocker toutes les solutions en
cours d'élaboration avant d'atteindre la solution.

\Question %
Modifiez votre parcours en largeur pour sauvegarder la liste de tous les états
visités afin de ne pas revisiter des sous-arbres déjà explorés. 
Une solution simple pour optimiser la consommation mémoire est de ne stocker la
nouvelle solution partielle que si elle n'est pas déjà présente dans la liste
des solutions partielles connues. Cette modification est très proche de celles
apportées lors de l'exercice précédent.

Sur machine, cette variante permet de trouver une solution à l'instance
Hounkonnou. Surprise, elle ne demande que 21 transvasements! En revanche, cela
ne dit pas quelle est cette solution\ldots Saurez-vous déterminer les opérations
nécessaires pour résoudre l'instance Hounkonnou?

\begin{Reponse}
  Sans l'optim, je trouve 1635380 états à la profondeur 8; avec l'optim, on
  découvre qu'il n'y a que 301 états distincts.
\end{Reponse}

\Question %
Testez votre code sur l'instance où l'on cherche 1 avec des récipients de taille
\{34, 55, 89, 144\}\footnote{Récipients=\{34, 55, 89, 144\}; Cible=1: Instance
  proposée par E. Thomé, chercheur en cryptographie au Loria.}. Cette instance
est plus complexe, et le code précédent ne suffit pas.

Saurez-vous déterminer le nombre minimal d'opération pour résoudre cette
instance? Quelles sont les opérations nécessaires pour la résoudre?

\Question %
Saurez-vous trouver une instance encore plus compliquée que celles connues à ce
jour? La complexité d'une instance du problème se mesure à la longueur de la
plus courte solution permettant de la résoudre.
\end{Exercice}


% \Exercice \textbf{Stockage par hachage.} Avec un peu plus de connaissances en
% Java que ce qui est demandé en TOP, une autre approche est d'utiliser une
% structure de données classique pour profiter des bonnes propriétés des tables
% de hachage (qui sont au programme du module de SD). Il suffit créer une
% variable de type \texttt{HashTree<String,Boolean>}, et d'utiliser ensuite les
% fonctions permettant de d'insérer un élément, et chercher si un élément donné
% existe dans la table. La clé des éléments sera le résultat de la méthode
% \texttt{toString()} appliquée à la solution courante. 

% \Question Implémentez cette solution en vous appuyant sur la documentation des
% \texttt{HashTree}.

% \Question Discutez l'efficacité de cette solution (en particulier en terme de
% mémoire).
% \begin{Reponse}
%   Ben c'est pas bon : on crée des chaines à tout bout de champs et on les
%   stoque. On va faire sauter la mémoire. 

%   En plus, on hache les chaines de représentation. On risque d'avoir pleins de
%   conflits vu que les chaines se ressemblent toutes. Mais ca, vu qu'ils ont pas
%   vu ce qu'est une table de hachage, on peut pas leur demander.
% \end{Reponse}

% \Exercice\textbf{Stockage par tableau booléen.} Avec un minimum de connaissances
% mathématiques, une autre approche est de coder chaque vecteur de remplissage
% sous forme d'un entier unique. Il faut pour cela trouver une fonction allant de
% l'ensemble des vecteurs possibles dans l'ensemble des entiers. On utilisera la
% fonction \texttt{int hashCode()} pour cela.

% \Question Quelle propriété doit avoir cette fonction?
% \begin{Reponse}
%   Injective (il me semble, je me goure tjs). Il faut que $\forall x,y, (x\neq y)
%   \Rightarrow (code(x)\neq code(y))$
% \end{Reponse}

% Pour construire une telle fonction, on peut par exemple multiplier chaque
% élément du vecteur par un nombre premier différent.  Par exemple, si le vecteur
% est de longueur 3, on peut utiliser les nombres premiers suivants:
% \{3,5,7\}. Ainsi, le vecteur [1,14,4] sera représenté par l'entier $3\times
% 1+14\times 5+7\times 4=101$. Il suffit alors de disposer d'un tableau booléen
% nommé par exemple \texttt{dejaVu}, et de stocker sous \texttt{dejaVu[101]} si
% l'on a déjà vu le vecteur [1,14,4].

% \Question Implémenter cette solution.

% \Question Discutez l'efficacité de cette solution. Comment peut-on améliorer
% les choses?
% \begin{Reponse}
%   Le hachage va mieux se passer, et on a beaucoup moins de chaînes
%   construites. C'est bien mieux. Mais on se trimbale maintenant avec un tableau
%   de booléens de dimension de folie.

%   On peut améliorer en stockant l'information sous forme d'une liste (chainée)
%   triée des valeurs déjà rencontrées. Le stockage sera plus efficace, la
%   recherche est en $O(\log(n))$, et l'insertion en $0(1)$.

%   Le mieux est de faire un stockage creux du vecteur (c'est comme ca qu'on dit
%   en calcul scientifique, rien de grave). On économise des pointeurs dans la
%   liste ci-dessus en stockant des vecteurs dans chaque case du tableau, et en
%   fusionnant les cellules adjacentes dans le même vecteur.

%   $(1) \leftrightarrow (2) \leftrightarrow (3) \leftrightarrow (76)
%   \leftrightarrow (77) \leftrightarrow 205$ devient $(1,2,3) \leftrightarrow
%   (76,77) \leftrightarrow 205$, ce qui sauve une poignée de vecteurs. Ca risque
%   d'être précieux si la structure grandit.
% \end{Reponse}

\end{document}
%%% Local Variables:
%%% coding: utf-8
